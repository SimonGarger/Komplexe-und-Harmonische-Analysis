\documentclass[11pt]{article}
\title{Zusammenfassung Komplexe und Harmonische Analysis 2023 WS}
\author{Simon Garger}
\usepackage{amsmath}
\usepackage{amsfonts}
\usepackage{setspace}
\usepackage{fullpage}
\usepackage{blkarray}
\usepackage[a4paper,left=2.5cm,right=2.5cm,top=2cm,bottom=4cm,bindingoffset=5mm]{geometry}
\usepackage{graphicx}
\usepackage{wasysym}
\usepackage{mathtools}
\usepackage{titlesec}
\usepackage{xcolor}
\usepackage{bm}
\usepackage{enumitem}
\usepackage{amsbooka}
\usepackage{translit}
\usepackage{amstex}



\newenvironment{problem}[1]{
    \begin{trivlist}
        \item[\hskip \labelsep {\bfseries #1}] }{
    \end{trivlist}\normalshape
}

\newcommand{\N}{\mathbb{N}}
\newcommand{\Z}{\mathbb{Z}}
\newcommand{\R}{\mathbb{R}}
\newcommand{\C}{\mathbb{C}}
\newcommand{\E}{\mathbb{E}}
\newcommand{\mP}{\mathbb{P}}
\newcommand{\ker}{\text{ker}}
\newcommand{\im}{\text{im}}
\newcommand{\rg}{\text{rg}}
\newcommand{\Cov}{\normalfont\text{Cov}}
\newcommand{\Var}{\normalfont\text{Var}}
\newcommand{\cor}{\normalfont\text{cor}}

\begin{document}
    \maketitle
    \section{Preliminaries to Complex Analysis}
    \begin{problem}{Holomorphie}
        Wir nennen eine Funktion $f:\C\to \C$ holomorph am Punkt $z$, wenn
        $$\lim_{h\to 0}\frac{f(z+h)-f(z)}{h}$$
        für $h\in \C$ existiert. \\\\
        Das ist genau dann der Fall, wenn für $f=u+iv$ $u$ und $v$ differenzierbar sind und sie
        die Cauchy-Riemann Gleichungen
        $$\frac{\partial u}{\partial x}=\frac{\partial v}{\partial y}\qquad
        \frac{\partial u}{\partial y}=-\frac{\partial v}{\partial x}$$
        erfüllen. \\\\
        Weiter definiert jede Potenzreihe eine holomorphe Funktion auf dem Konvergenzradius und
        die termweise Ableitung entspricht der Ableitung selbst. Damit ist eine holomorphe Funktion
        unendlich oft differenzierbar und die Ableitung hat den gleichen Konvergenzradius.
    \end{problem}

    \section{Cauchy’s Theorem and Its Applications}
    \begin{problem}{Goursat}
        $\Omega$ offen und $T\subset\Omega$ ein Dreieck, wessen Inneres in $\Omega$ liegt, dann
        gilt
        $$\int_T f(z)dz=0$$
        wenn $f$ holomorph auf $\Omega$ ist.
    \end{problem}

    \begin{problem}{Stammfunktionen}
        Eine holomorphe Funktion hat auf jeder offenen Kreisscheibe eine Stammfunktion. Damit ist
        das Integral holomorpher Funktionen über geschlossene Kurven immer $0$.
    \end{problem}

    \begin{problem}{Cauchy-Integral-Formel}
        Ist $f$ holomorph auf einer offenen Menge, die den Abschluss einer Kreisscheibe $D$ enthält.
        Sei nun $C$ der Rand von $D$ in positiver Richtung dann gilt die Cauchy-Integral-Formel
        $$f(z)=\frac{1}{2\pi i}\int_C\frac{f(\zeta)}{\zeta -z}d\zeta$$
        für alle $z\in D$. \\\\
        Für die $n$-te Ableitung gilt entsprechend:
        $$f^{(n)}(z)=\frac{n!}{2\pi i}\int_C\frac{f(\zeta)}{(\zeta-z)^{n+1}}d\zeta$$
        Womit man auch die Abschätzung
        $$|f^{(n)}(z_0)|\leq \frac{n!\|f\|_C}{R^n}$$
    \end{problem}

    \begin{problem}{Liouville}
        Eine ganze und beschränkte Funktion $f$ ist konstant.
    \end{problem}

    \begin{problem}{Analytische Fortsetzung}
        Gilt für $f,g$ holomorph auf $\Omega$ auch $f(z)=g(z)$ auf einer Folge unterschiedlicher Punkte mit
        Grenzwert in $\Omega$, dann sind $f$ und $g$ gleich.
    \end{problem}

    \begin{problem}{Morera}
        Eine stetige Funktion, die auf einer offenen Teilmenge für jedes Dreick im Inneren
        $$\int_T f(z)dz=$$
        erfüllt, ist holomorph.
    \end{problem}

    \begin{problem}{Folgen holomorpher Funktionen}
        Der Grenzwert $f$ einer Folge holomorpher Funktionen $f_n$ ist holomorph, wenn die
        Folge gleichmäßig konvergiert.
    \end{problem}

    \begin{problem}{Schwarz reflection principle}
        Sei $\Omega$ eine offene Menge symmetrisch um die reelle Achse und sei weiter $f$ eine
        holomorphe Funktion auf dem "positiven Teil" $\Omega^+$. Kann die Funktion weiter
        stetig auf die reelle Achse fortgesetzt werden, sodass sie dort nur reelle Werte annimmt, dann
        kann man die Funktion gleich auf $\Omega^-$ fortsetzen. Das passiert durch $F=f$ auf $\Omega^+$ und
        $F(z)=\overline{f(\overline{z})}$ für $z\in\Omega^-$.
    \end{problem}

    \begin{problem}{Runge's approximation theorem}
        Eine holomorphe Funktion kann in der Umgebung von einer kompakten Menge $K$ gleichmäßig
        auf $K$ durch rationale Funktionen approximiert werden. Die Singularitäten diese Funktionen liegen
        in $K^c$. \\
        Ist $K^c$ zusammenhängend, dann kann man die Funktion sogar durch Polynome approximieren.
    \end{problem}
    
    \section{Meromorphic Functions and the Logarithm}
    \begin{problem}{Singularities}
        Wir haben drei verschiedene Arten an Singularitäten:
        \begin{itemize}
            \item Hebbare Singularität ($f$ beschränkt)
            \item Pol ($1/f$ beschränkt, hat eine Ordnung $n$)
            \item Essentielle Singularität ($f(B_\varepsilon(z_0))$ liegt dicht in $\C$)
        \end{itemize}
    \end{problem}

    \begin{problem}{Residuum}
        Hat $f$ einen Pol der Ordnung $n$ bei $z_0$, dann können wir
        $$f(z)=\frac{a_{-n}}{(z-z_0)^n}+\cdots + \frac{a_{-1}}{(z-z_0)}+G(z)$$
        mit $G$ holomorph und der Teil davor heißt Hauptteil (principal part). Weiter haben wir
        $\text{res}_{z_0} f=a_{-1}$. \\\\
        Wir bestimmen das Residuum im Allgemeinen mit
        $$\text{res}_{z_0} f=\lim_{z\to z_0}\frac{1}{(n-1)!}\left(\frac{d}{dz}\right)^{n-1}(z-z_0)^n f(z)$$
        und für $n=1$ einfach mit $\text{res}_{z_0} f=\lim_{z\to z_0}(z-z_0)f(z)$.
    \end{problem}

    \begin{problem}{Residuen Formel}
        Ist $f$ holomorph auf einer offenen Menge, welche den Kreis (Contour Integral) $C$ und sein Inneres
        enthält bis auf die Pole $z_1,\dots, z_N$ im Kreis. Dann gilt
        $$\int_C f(z)dz = 2\pi i\sum_{k=1}^N \text{res}_{z_k} f$$
    \end{problem}
    
    \begin{problem}{Argument principle}
        Sei $f$ meromorph in einer offenen Menge, die einen Kreis (Contourintegral)
        $C$ und sein Inneres enthält. Wenn $f$ keine Pole und keine Nullstellen
        auf $C$ hat, dann gilt 
        $$\frac{1}{2\pi i}\int_C \frac{f'(z)}{f(z)}dz=\text{ (Nullstellen innerhalb
            von $C$) } - \text{ (Anzahl an Polen innerhalb von C)}$$
    \end{problem}

    \begin{problem}{Rouché's Theorem}
        Seien $f$ und $g$ holomorph in einer offenen Menge, die einen Kreis $C$
        und sein Inneres enthält. Wenn nun $|f(z)|>|g(z)|$ auf $C$ gilt,
        dann haben $f$ und $f+g$ gleich viele Nullstellen innerhalb von $C$.
    \end{problem}

    \begin{problem}{Folgerungen}
        Eine nichtkonstante, holomorphe Funktion $f$ bildet offene Mengen
        auf offene Mengen ab. \\\\
        Eine nichtkonstante, holomorphe Funktion nimmt auf einer offenen Menge
        $\Omega$ nicht ihr Maximum an.
    \end{problem}

    \begin{problem}{Homotopie}
        Zwei Kurven $\gamma_0$, $\gamma_1$ sind homotop, wenn sie die gleichen
        Anfangs- und Endpunkte haben und für jedes $0\leq s\leq 1$ eine
        Kurve existiert $\gamma_s$ existiert, die ebenso die selben Anfangspunkte
        hat und in $s$ stetig ist. Es geht also lasch gesagt darum, dass man
        die eine Kurve stetig zu der anderen umformen kann. \\\\
        Für holomorphe Funktionen $f$ und homotope Kurven $\gamma_0$, $\gamma_1$
        gilt
        $$\int_{\gamma_0}f(z)dz=\int_{\gamma_1}f(z)dz$$
        In einfach zusammenhängenden Gebieten sind alle Kurven mit gleichen
        Endpunkten homotop. Auf einfach zusammenhängenden Gebieten
        haben deswegen auch alle holomorphen Funktionen eine Stammfunktion.
    \end{problem}

    \begin{problem}{Komplexer Logarithmus}
        Weil die Umkehrung von $e^z$ nicht aufgrund des Phasenwinkels von $z$
        nicht eindeutig ist, können wir den Logarithmus nicht global definieren
        sonderen nur auf Branches/Zweigen, also der komplexen Zahlenebene mit
        einem Schlitz. \\\\
        Diese Gebiete sind dann nämlich einfach zusammenhängend (weil wir die
        $0$ ausnehmen). Auf diesen definieren wir nun:
        $$\log(z)=\log(r)+i\theta$$
        wobei $z=re^{i\theta}$. Dieser Logarithmus erbt im Allgemeinen nicht
        die Rechenregeln vom reellen Logarithmus. Wir können aber damit für
        alle $\alpha$
        $$z^{\alpha}=e^{\alpha\log(z)}$$
        definieren.
    \end{problem}

    \section{Fourier Reihe}
    \begin{problem}{Fourier Reihe}
        Für eine integrierbare Funktion auf dem Intervall $[-\pi,\pi]$ definieren wir
        den $n$-ten Fourierkoeffizienten
        $$\hat{f}(n)=\frac{1}{2\pi}\int_{-\pi}^{\pi} f(\theta)e^{-in\theta}d\theta, \qquad n\in\Z$$
        und die Fourierreihe durch
        $$f(x)\sim \sum_{n=-\infty}^\infty \hat{f}(n) e^{inx}$$
        wobei die letzte Reihe über den symmetrischen Grenzwert definiert ist.
    \end{problem}

    \begin{problem}{Eindeutigkeit}
        Sei $f:\R\to\C$ $2\pi$-periodisch und integrierbar mit
        $$\hat{f}(n)=0,\qquad n\in\Z$$
        und $x_0\in \R$. Wenn $f$ bei $x_0$ stetig ist, dann $f(x_0)=0$. Sonst ist die Funktion f.ü. $0$\\\\
        Damit folgt sofort
        Sei $f,g:\R\to\C$ $2\pi$-periodisch und stetig mit
        $$\hat{f}(n)=\hat{g}(n),\qquad n\in\Z$$
        Dann gilt $f\equiv g$.\\\\
        Weiter ist für eine periodische und stetige Funktion mit
        $\sum_{n\in\Z}|\hat{f}(n)|<\infty$. Dann
        $$f(x)=\sum_{n\in\Z}\hat{f}(n)e^{inx},\qquad x\in\R$$
        mit absoluter und gleichmäßiger Konvergenz.\\\\
        Eine weitere Folgerung ist, dass für eine periodische und integrierbare Funktion $f$ die auch
        $\sum_{n\in\Z}|\hat{f}(n)|<\infty$ erfüllt gilt, dass $f$ fast stetig ist, also existiert $g$ stetig
        mit $f=g$ f.ü.
    \end{problem}

    \begin{problem}{Absolute Konvergenz der Fourierkoeffizienten}
        Für $f:\R\to\C$ $2\pi$-periodisch und stetig differenzierbar gilt
        $$\hat{f'}(n)=in\hat{f}(n)$$
        und damit konvergiert die Fourier-Reihe auf jeden Fall absolut und gleichmäßig gegen $f$, wenn
        die Funktion zweimal stetig differenzierbar ist. Außerdem erhalten wir die Abschätzung
        $$|\hat{f}(n)|\leq C|n|^{-2},\qquad n\in\Z\setminus\{0\}$$
    \end{problem}

    \begin{problem}{Faltungsprodukt}
        Für $f,g$ periodisch, $f$ integrierbar und $g$ messbar und beschränkt definieren wir
        die Faltung von $f$ und $g$ als
        $$(f*g)(x)=\frac{1}{2\pi}\int_{-\pi}^{\pi} f(y)g(x-y)dy,\qquad x\in\R$$
        Dieses ist auch periodisch und stetig. Weiter gilt
        $$\hat{f*g}(n)=\hat{f}(n)\hat{g}(n)$$
        und wir haben alle Eigenschaften von Produkten, Assoziativität mit Skalaren, Distributivität,
        Kommutativität, Assoziativität.
    \end{problem}

    \begin{problem}{Folge guter Kerne}
        Eine Folge $(K_n)_{n\geq 1}$ heißt gut, wenn
        \begin{enumerate}[label = (\roman{enumi})]
        \item Für alle $n\geq 1$ gilt
        $$\frac{1}{2\pi}\int_{-\pi}^{\pi} K_n(x)dx=1$$
        \item Es gibt ein $M>0$, so dass für alle $n\geq 1$
        $$\int_{-\pi}^{\pi} |K_n(x)|dx\leq M$$
        \item Für jedes $\delta>0$ gilt
        $$\int_{\delta\leq |x|\leq \pi}|K_n(x)|dx\to 0 $$
        für $n\to\infty$.
        \end{enumerate}
        In diesem Fall gilt $(f*K_n)(x_0)\to f(x_0)$, wenn $f$ auf $x_0$ stetig ist und wenn $f$ stetig
        ist, dann $f*K_n\to f$ gleichmäßig. Für $f$ periodisch und integrierbar gilt immernoch
        $$\|f*K_n-f\|_1=\frac{1}{2\pi}\int_{-\pi}^{\pi}|(f*K_n-f)(x)|dx$$
    \end{problem}

    \begin{problem}{Kerne}
        Der Dirichlet Kern ist definiert durch
        $$D_N(x):=\sum_{n=-N}^N e^{inx}=\frac{\sin\left(\frac{N+1}{2}\cdot x\right)}{\sin\left(
        \frac{x}{2}\right)}$$
        Ist keine Folge guter Kerne, die Faltung konvergiert aber trotzdem.\\\\
        Der Poisson Kern ist für $\in (0,1)$ definiert
        $$P_r(x):=\sum_{n\in \Z}r^{|n|}e^{inx}=\frac{1-r^2}{1-2r\cos(x)+r^2}$$
        Ist eine Folge guter Kerne. Vorallem gilt für die Abel-Durchschnitte
        $$A_r(f)(x):=\sum_{n\in\Z}r^{|n|}\hat{f}(n)e^{inx}=(f*P_r)(x)\to f,\qquad r\in(0,1)$$
        wobei die letzte Konvergenz gleichmäßig ist. \\\\
        Der Fejér Kern ist definiert als
        $$F_N(x):=\frac{1}{N}\sum_{M=0}^{N-1}D_M(x)=\frac{1}{N}\frac{\sin^2(Nx/2)}{\sin^2(x/2)}$$
        Es ist eine gute Folge von Kernen. Vorallem gilt für die Cesàro-Durchschnitte
        $$C_n(f)(x):= \frac{1}{N}\sum_{M=0}^{N-1}\sum_{n=-M}^M \hat{f}(n)e^{inx}=(f*F_n)(x)
        =\sum_{n=-(N-1)}^{N-1}(1-\frac{|n|}{N})\hat{f}(n)e^{inx}\to f$$
        wobei die letzte Konvergenz gleichmäßig ist.
    \end{problem}

    \begin{problem}{Interpretation als Hilbertraum}
        Wir können den $L^2([-\pi,\pi])$ (quadratintegrierbare Funktion mit Äquivalenzklassen auf f.ü.
        übereinstimmenden Funktionen) als Hilbertraum mit dem Skalarprodukt
        $$\langle f,g \rangle:= \frac{1}{2\pi}\int_{-\pi}^{\pi} f(x)g^*(x)dx$$
        Dabei bildet $e^{inx}$ mit $n\in\Z$ eine Orthonormalbasis und somit finden wir mit
        $S_N(f)(x):=\sum_{n=-N}^N \hat{f}(n)e^{inx}$ die quadratische Konvergenz
        $$\frac{1}{2\pi}\int_{-\pi}^\pi |f(x)-S_N(f)(x)|^2 dx\to 0$$
        und die Parseval-Isometrie
        $$\frac{1}{2\pi}\int_{-\pi}^{\pi}|f(x)|^2dx =\sum_{n\in \Z}|\hat{f}(n)|^2$$
    \end{problem}

    \begin{problem}{Lipschitz}
        Sei $f$ periodische, integrierbar und lokal Lipschitz bei $x_0$, dann gilt
        $$S_N(f)(x_0)\to f(x_0)$$
        für $N\to\infty$
    \end{problem}

    \begin{problem}{Lokalisierungsprinzip von Riemann}
        Seien $f,g$ periodisch, messbar und beschränkt. Nehmen wir an $f\equiv g$ auf einer Umgebung von
        $x_0$. Dann $S_N(f)(x_0)\to f(x_0)$ genau dann wenn $S_N(g)(x_0)\to f(x_0)$
    \end{problem}

    \begin{problem}{Fourier-Transformation}
        Sei $f\in L^1 (\R)$. Die Fourier-Transformation von $f$ ist
        $$\hat{f}(\xi) = \int_\R f(x)e^{-2\pi i x\xi}dx,\qquad \xi \in\R$$
        Die Abbildung $L^1(\R)\to L^{\infty}(\R), f\mapsto \hat{f}$ ist
        stetig und linear.\\\\
    \end{problem}

    \begin{problem}{Schwartz-Funktionen}
        Eine Funktion $f$ heißt Schwartz, wenn sie unendlich oft differenzierbar
        ist und für alle $k,j\in\N$ auch
        $$\sup_{x\in \R} (1+|x|)^k \left|\frac{d^j}{dx^j}f(x)\right|<\infty$$
        Ist eine Funktion Schwartz so sind es auch alle Ableitungen
        und alle Multiplikationen mit Polynomen. Wichtigstes Beispiel
        ist die Gauß-Funktion $\Phi(x) = e^{-\pi x^2}$. \\\\
        Weiter gilt für $f\in\mathcal{S}(\R)$ und $g\in L^1(\R)$ mit
        $$\int_\R (1+|x|)^k|g(x)|dx <\infty$$
        dann ist auch die Faltung $f*g\in \mathcal{S}(\R)$. \\\\
        Sind $f,g\in\mathcal{S}(\R)$, dann gilt
        $$\widehat{f*g}(\xi) = \hat{f}(\xi)\hat{g}(\xi),\qquad \xi\in \R$$
    \end{problem}

    \begin{problem}{Familie von guten Kerne}
        Eine Familie $K_\varepsilon:\R\to\C$ heißt gut, wenn
        \begin{enumerate}[label = (\roman{enumi})]
            \item
            $$\int_{\R} K_\varepsilon(x)dx=1$$
            \item
            $$\sup_{\varepsilon >0}\int_{\R} |K_\varepsilon(x)|dx<\infty$$
            \item Für jedes $\delta>0$ gilt
            $$\int_{\delta< |x|}|K_\varepsilon(x)|dx\to 0 $$
            für $\varepsilon\to 0^+$.
        \end{enumerate}
        Aus einer Funktion $K\in L^1(\R)$ mit $\int_\R K(x)dx = 1$ gewinnt
        man eine solche Familie durch $K_\varepsilon:= \frac{1}{\varepsilon}
        K(x/\varepsilon)$.
    \end{problem}

    \begin{problem}{Eigenschaften der Fouriertransofrmation}
        Für $f,g\in L^1(\R)$ gilt:
        $$\int_{\R} f(x)\hat{g}(x) dx = \int_\R \hat{f}(y) g(y)dy$$
        Für $f,\hat{f}\in L^1(\R)$ gilt damit
        $$f(x) = \int_\R \hat{f}(\xi) e^{2\pi i x\xi} d\xi$$
        für fast jedes $x\in \R$. Ist $f$ sogar Schwartz gilt
        es für jedes $x$. \\\\
        Nach Plancherel gilt für $f\in\mathcal{S}(\R)$
        $$\int_{\R}|f(x)|^2 dx=\int_{\R}|\hat{f}(x)|^2 d\xi $$
    \end{problem}

    \begin{problem}
        Je konzentrierter eine Funktion ist, desto breiter ist die
        Fourier-Transformation. \\\\
        Wir messen die Verbreitung mit:
        $$V(f) = \frac{\int_\R x^2 |f(x)|^2 dx}{\int_\R}|f(x)|^2 dx$$
        Anwendung davon ist die Heisenbergsche Unschärferelation. 
    \end{problem}

    Fehler Skript: Beweis Satz 4 vorletzte Zeile, zweimal a_n 
\end{document}