\documentclass[11pt]{article}
\title{Zusammenfassung Numerik 2022 WS}
\author{Simon Garger}
\usepackage{amsmath}
\usepackage{amsfonts}
\usepackage{setspace}
\usepackage{fullpage}
\usepackage{blkarray}
\usepackage[a4paper,left=2.5cm,right=2.5cm,top=2cm,bottom=4cm,bindingoffset=5mm]{geometry}
\usepackage{graphicx}
\usepackage{wasysym}
\usepackage{mathtools}
\usepackage{titlesec}
\usepackage{xcolor}
\usepackage{bm}
\usepackage{enumitem}
\usepackage{amsbooka}
\usepackage{translit}
\usepackage{arabicore}
\usepackage{amsmath-2018-12-01}


\newcommand{\N}{\mathbb{N}}
\newcommand{\Z}{\mathbb{Z}}
\newcommand{\R}{\mathbb{R}}
\newcommand{\C}{\mathbb{C}}
\renewcommand{\Re}{\mathfrak{Re}}
\renewcommand{\Im}{\mathfrak{Im}}


\newenvironment{problem}[2][Beispiel]{
    \begin{trivlist}
        \item[\hskip \labelsep {\bfseries #1}\hskip \labelsep {\bfseries #2.}] \itshape}{
    \end{trivlist}\normalshape
}

\begin{document}
    \begin{problem}{1}
        Benutze Kurvenintegrale in der komplexen Ebene, um das Integral
        $$\int_0^{\infty}\frac{\sin(x)}{x}\,dx$$
        zu berechnen.
    \end{problem}

    \begin{proof}
        Wir wählen die Funktion $f(z) = \frac{e^{iz}}{z}$.\\
        Wir berechnen das Integral einer geschlossenen Kurve, die aus einem
        großen Halbkreis mit Radius $R$, zwei Geraden Strecken von
        $-R$ bis $-\varepsilon$ bzw. $\varepsilon$ bis $R$ und einem kleinen
        Halbkreis mit Radius $\varepsilon$. Dafür gilt:
        $$0 = \int_{\gamma_R}f(z)dz+\int_{-R}^{-\varepsilon} f(x)dx
        +\int_{\varepsilon}^R f(x)dx + \int_{\gamma_{\varepsilon}}f(z)dz$$
        Wir bestimmen die einzelnen Integrale und fangen mit den geraden Strecken an:
        $$\begin{aligned}
              \int_{-R}^{-\varepsilon} f(x)dx +\int_{\varepsilon}^R f(x)dx &=
              \int_{-R}^{-\varepsilon} \frac{e^{ix}}{x}dx +\int_{\varepsilon}^R
              \frac{e^{ix}}{x}dx\\
              &= \int_{R}^{\varepsilon} \frac{e^{-iu}}{-u}-du +\int_{\varepsilon}^R
              \frac{e^{ix}}{x}dx\\
              &= -\int_{\varepsilon}^R \frac{e^{-iu}}{u}du +\int_{\varepsilon}^R
              \frac{e^{ix}}{x}dx\\
              &= \int_{\varepsilon}^R \frac{e^{ix}-e^{-ix}}{x}dx\\
              &= 2i\int_{\varepsilon}^R \frac{\sin(x)}{x}dx\\
        \end{aligned} $$
        Die Wahl unserer Funktion war als sinnvoll, weil so das Integral der beiden
        Strecken annähernd dem gesuchten Integral entspricht.\\\\
        Nun berechnen wir das Integral über den großen Halbkreis und parametrisieren
        diesen dafür mit $\gamma_R: t\mapsto Re^{it}, t\in [0,\pi]$. Damit gilt:
        $$\int_{\gamma_R}f(z)dz = \int_0^\pi \frac{e^{iRe^{it}}}{Re^{it}} iRe^{it}dt =
        i\int_0^\pi e^{iR (\cos(t)+ i \sin(t))}dt =
        i\int_0^\pi e^{iR\cos(t)} e^{-R\sin(t)}dt$$
        Nun gilt $|e^{R\cos(t)}| = 1$ und $e^{-R\sin(t)} \to 0$ für $R\to \infty$ f.s.
        Also ergibt das obige Integral den Wert $0$, wenn $R$ gegen $\infty$ geht.
        \\\\
        Nun berechnen wir noch das Integral über den kleinen Halbkreis.
        Dafür parametrisieren wir wie folgt: $\gamma_{\varepsilon}: t \mapsto
        \varepsilon e^{it}, t\in[\pi, 0]$ und finden damit:
        $$\int_{\gamma_{\varepsilon}} \frac{e^{i\varepsilon e^{it}}}{\varepsilon e^{it}}
        i\varepsilon e^{it}dt = i\int_{\pi}^0 e^{i\varepsilon e^{it}} dt \to
        -i\pi
        $$
        für $\varepsilon\to 0$. Nun gilt also gesamt (für $R\to\infty$ und
        $\varepsilon\to 0$):
        $$0 = 2i \int_0^{\infty}\frac{\sin(x)}{x}\,dx - i\pi
        \Rightarrow \int_0^{\infty}\frac{\sin(x)}{x}\,dx = \frac{\pi}{2}$$
    \end{proof}

    \begin{problem}{2}
        Zeige, dass
        $$\int_0^\infty \sin(x^2)\,dx = \int_0^\infty\cos(x^2)\,dx = \frac{\sqrt{2\pi}}{4}.$$
        (Hinweis: Betrachte $z\mapsto e^{-z^2}$ auf einem Kreissektor mit Winkel $\frac{\pi}{4}$.)
    \end{problem}

    \begin{proof}
        Wir legen den Kreissektor entlang der $x$-Achse auf und erhalten somit drei Teilkurven,
        die zusammen eine geschlossene Kurve ergeben:
        $$\gamma_1: t\mapsto t, t\in [0,R];\qquad \gamma_2: t\mapsto Re^{it}, t\in \left[0,\frac{\pi}{4}\right];\qquad
        \gamma_3: t\mapsto te^{i\frac{\pi}{4}}, t\in [R,0]$$
        Wir berechnen die Integrale von $e^{-z^2}$ über diese Kurven und finden damit:
        $$\int_{\gamma_1} e^{-z^2}dz = \int_{0}^R e^{-t^2}dt\to \frac{\sqrt{\pi}}{2}$$
        Für $R\to\infty$.
        $$\int_{\gamma_2} e^{-z^2}dz = \int_0^{\pi/4} e^{- R^2 e^{i2t}}iRe^{it}dt =
        iR\int_0^{\pi/4}e^{- R^2\cos(2t)}e^{- iR^2\sin(2t)}e^{it}$$
        Betragsmäßig ist diese Integral kleiner gleich $\frac{\pi}{4} Re^{-R^2}\to 0$ für $R\to\infty$.
        $$\int_{\gamma_3} e^{-z^2}dz = -\int_0^{R} e^{- t^2 e^{i\frac{\pi}{2}}}e^{i\frac{\pi}{4}}dt
        = -e^{i\frac{\pi}{4}}\int_0^{R} e^{-it^2}dt = -e^{i\frac{\pi}{4}}\left(\int_0^R\cos(-t^2)dt +i\int_0^R\sin(-t^2)\right)$$
        Wir nutzen nun, dass der Cosinus gerade und der Sinus ungerade ist. Anschließend setzen wir
        zusammen:
        $$\frac{\sqrt{\pi}}{2}e^{-i\frac{\pi}{4}} = \frac{\sqrt{\pi}}{2}\left(\frac{1}{\sqrt{2}} - i
        \frac{1}{\sqrt{2}}\right)=\int_0^R\cos(t^2)dt -i\int_0^R\sin(t^2)$$
        Durch vergleichen von Real- und Imaginärteil auf beiden Seiten folgt das Behauptete.
    \end{proof}

    \begin{problem}{3}
        Sei $f: B_1(0)\to\C$ holomorph. Zeige: Dann gilt
        $$|f'(0)|\leq \frac{1}{2}\sup_{z,w\in B_1(0)} |f(w)-f(z)|.$$
    \end{problem}

    \begin{proof}
        Zuerst finden wir mit der Cauchy-Integralformel:
        $$\begin{aligned}
              2 f'(0) &= \frac{1}{2\pi i}\left(\int_{\partial B_1(0)}\frac{f(\zeta)}{\zeta^2}d\zeta
              +\int_{\partial B_1(0)}\frac{f(\xi)}{\xi^2}d\xi\right)\\
              &= \frac{1}{2\pi i}\left(\int_{\partial B_1(0)}\frac{f(\zeta)}{\zeta^2}d\zeta
              -\int_{\partial B_1(0)}\frac{f(-\zeta)}{\zeta^2}d\zeta\right)\\
              &= \frac{1}{2\pi i}\left(\int_{\partial B_1(0)}\frac{f(\zeta)}{\zeta^2}d\zeta
              -\int_{\partial B_1(0)}\frac{f(-\zeta)}{\zeta^2}d\zeta\right)\\
              &= \frac{1}{2\pi i}\int_{\partial B_1(0)}\frac{f(\zeta)-f(-\zeta)}{\zeta^2}d\zeta
        \end{aligned}$$
        Nun können wir die gleichen Abschätzungen wie bei den Cauchy-Ungleichungen durchführen (für $R<1$):
        $$\begin{aligned}
              2|f'(0)|
              &= \left|\frac{1}{2\pi i}\int_{\partial B_1(0)}\frac{f(\zeta)-f(-\zeta)}{\zeta^2}d\zeta\right|\\
              &= \frac{1}{2\pi}\left|\int_{0}^{2\pi}\frac{f(Re^{it})-f(-Re^{it})}{(Re^{it})^2}iRe^{it}dt\right|\\
              &\leq \frac{1}{2\pi R}\int_{0}^{2\pi}\left|\frac{f(Re^{it})-f(-Re^{it})}{(Re^{it})^2}\right|dt\\
              &\leq \frac{1}{2\pi R}\cdot 2\pi \cdot \sup_{w,z\in B_1(0)}|f(w)-f(z)| = \frac{1}{R}\sup_{w,z\in B_1(0)}|f(w)-f(z)|
        \end{aligned}$$
        Wäre nun $2|f'(0)|>\sup_{w,z\in B_1(0)}|f(w)-f(z)|$, könnten wir ein $\varepsilon>0$ finden, sodass
        $2|f'(0)|-\varepsilon>\sup_{w,z\in B_1(0)}|f(w)-f(z)|$ auch gilt. Diesen Fall können wir
        aber mit der oben erhaltenen Ungleichung zum Widerspruch führen, indem wir
        $R = 1- \frac{\varepsilon}{2|f'(0)|}$ wählen und oben einsetzen.
    \end{proof}

    \begin{problem}{4}
        Zeige oder widerlege: Für jedes stetige $f: \overline{B_1(0)}\to\C$ gibt es eine Folge von
        komplexen Polynomen $P_n(z)$ die gleichmäßig gegen $f$ konvergieren wenn $n\to\infty$.\\
        (Zum Kontext: Jede reelle, stetige Funktion $g : [0, 1] \to \R$ auf dem Einheitsintervall kann
        gleichmässig durch Polynome approximiert werden (Satz von Weierstrass).)
    \end{problem}

    \begin{proof}
        Betrachten wir die Funktion $f(z) = \bar{z}$ und eine beliebige Folge $P_n(z)$ von Polynomen, die
        gleichmäßig gegen $P(z)$ konvergiert. Da Polynome holomorphe Funktionen sind, konvergieren
        diese nach Satz 5.2 gleichmäßig auf jeder kompakten Teilmege gegen eine holomorphe Funktion.
        Somit ist $P$ holomorph und damit $P\neq f$. Da die Folge beliebig war, ist die Aussage widerlegt.
    \end{proof}

    \begin{problem}{5}
        Sei $f:\C\to\C$ ganz. Zeige: Falls es ein $M>0$ gibt so dass $\Re(f(z))\leq M$ für alle $z\in\C$,
        dann ist $f$ konstant.
    \end{problem}

    \begin{proof}
        Da $f$ ganz ist, ist auch $e^{f(z)}$ ganz und wir finden:
        $$|e^{f(z)}| = |e^{u(z)+iv(z)}| = |e^{u(z)}e^{iv(z)}| = |e^{\Re(f(z))}|\leq e^M $$
        Nach Liouville ist also $e^{f(z)}$ konstant, das ist wiederrum äquivalent mit $f(z)$ konstant.
    \end{proof}

\end{document}