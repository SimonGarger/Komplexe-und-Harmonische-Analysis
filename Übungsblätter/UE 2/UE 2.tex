\documentclass[11pt]{article}
\title{Zusammenfassung Numerik 2022 WS}
\author{Simon Garger}
\usepackage{amsmath}
\usepackage{amsfonts}
\usepackage{setspace}
\usepackage{fullpage}
\usepackage{blkarray}
\usepackage[a4paper,left=2.5cm,right=2.5cm,top=2cm,bottom=4cm,bindingoffset=5mm]{geometry}
\usepackage{graphicx}
\usepackage{wasysym}
\usepackage{mathtools}
\usepackage{titlesec}
\usepackage{xcolor}
\usepackage{bm}
\usepackage{enumitem}
\usepackage{amsbooka}
\usepackage{translit}
\usepackage{arabicore}


\newcommand{\N}{\mathbb{N}}
\newcommand{\Z}{\mathbb{Z}}
\newcommand{\R}{\mathbb{R}}
\newcommand{\C}{\mathbb{C}}
\renewcommand{\Re}{\mathfrak{Re}}
\renewcommand{\Im}{\mathfrak{Im}}


\newenvironment{problem}[2][Beispiel]{
    \begin{trivlist}
        \item[\hskip \labelsep {\bfseries #1}\hskip \labelsep {\bfseries #2.}] \itshape}{
    \end{trivlist}\normalshape
}

\begin{document}
    \begin{problem}{1}
        Welche der folgenden Aussagen sind richtig bzw. falsch? Begründe:
        \begin{enumerate}[label = (\alph*)]
            \item Die Abbildung $f_1: \mathbb{C} \rightarrow \mathbb{C}, z=x+i y \mapsto x^5 y^4-i x^4 y^5$ ist
            holomorph.
            \item Die Abbildung $f_2: \mathbb{C} \rightarrow \mathbb{C}, z=x+i y \mapsto e^y-i e^x$
            ist holomorph.
            \item Die Abbildung $f_3: \mathbb{C} \rightarrow \mathbb{C}, z
            \mapsto \sin \left(z^4|z|^2\right)$ ist holomorph.
            \item Falls $f: \mathbb{C} \rightarrow \mathbb{C}$ holomorph ist, so sind dies auch
            $g_1(z):=\overline{f(z)}, g_2(z):=f(\bar{z})$ und $g_3(z):=\overline{f(\bar{z})}$.
        \end{enumerate}
    \end{problem}

    \begin{proof}
        Wir finden:
        \begin{enumerate}[label = (\alph*)]
            \item Wir prüfen die Cauchy-Riemann Gleichungen, dabei gilt $u(x+iy)=x^{5}y^4, v(x+iy) = -x^{4}y^5$:
            $$\frac{\partial u}{\partial x} = 5x^4y^4\neq -5x^4 y^4 = \frac{\partial v}{\partial y}$$
            Also ist die Funktion nicht holomorph.
            \item Wir prüfen die Cauchy-Riemann Gleichungen, dabei gilt $u(x+iy)=e^y, v(x+iy) = -e^x$:
            $$\frac{\partial u}{\partial y} = e^y\neq e^x = -\frac{\partial v}{\partial x}$$
            Also ist die Funktion nicht holomorph.
            \item Wäre die Funktion holomorph auf $\C$ so müsste ihr Integral über eine geschlossene Kurve 
            verschwinden. Wir betrachten hierfür die Kurven
            $$\gamma_1: (0,\pi)\to\C, t\mapsto e^{it}, \gamma_2: [-1,1]\to\C, t\mapsto t$$
            und finden damit:
            $$\begin{aligned}
                  \int_{\gamma_1}\sin \left(z^4|z|^2\right)\,dz +\int_{\gamma_2}\sin \left(z^4|z|^2\right)\,dz &=
                  \int_{0}^{\pi} \sin \left(e^{4it}\right)ie^{it}\,dt + \underbrace{\int_{-1}^1 \sin(t^6),dt}_{=:r\in\R}
                  \\&= i\int_{0}^{\pi} \frac{e^{4it}-e^{-4it}}{2i} e^{it}\,dt+r
                  \\&=\frac{1}{2}\int_{0}^{\pi}e^{5it}-e^{-3it}\,dt+r
                  \\&= \frac{1}{2}\left[\frac{1}{5i}e^{5it}+\frac{1}{3i}e^{-3it}\right]_0^{\pi}+r
                  \\&=\frac{1}{2}\left(-\frac{1}{5i}-\frac{1}{3i}-\frac{1}{5i}-\frac{1}{3i}\right)+r
                  \\&= r+ i\left(\frac{1}{5}+\frac{1}{3}\right)\neq 0
            \end{aligned}$$
            Also ist die Funktion nicht holomorph.\\\\
            Geht leichter mit Ableitung nach $\bar{z}$. Nochmal anschauen, nich ganz verstanden.
            \item $g_1(z)$ und $g_2(z)$ sind im Allgemeinen nicht holomorph, wie man an dem Beispiel $f(z) = z$ sieht.
            Für $g_3(z)$ hingegen finden wir:
            $$\begin{aligned}
                  &\lim_{h\to0} \frac{g_3(z+h)-g_3(z)}{h} = \lim_{h\to0}
                  \frac{\overline{f(\overline{z+h})}-\overline{f(\bar{z})}}{h} = \lim_{h\to0}
                  \frac{\overline{f(\bar{z}+\bar{h})-f(\bar{z})}}{h} \\&= \lim_{h\to0}
                  \overline{\left(\frac{f(\bar{z}+\bar{h})-f(\bar{z})}{\bar{h}}\right)} =
                  \overline{f'(\bar{z})}
            \end{aligned}$$
            Also ist die Funktion holomorph.
        \end{enumerate}
    \end{proof}

    \begin{problem}{2}
        Sei
        $$
        \alpha(t):= \begin{cases}1-e^{i t}, & t \in[0,2 \pi], \\ -1+e^{-i t}, & t \in[2 \pi, 4 \pi] .
        \end{cases}
        $$
        Skizziere die durch $\alpha$ parametrisierte Kurve und berechne das Kurvenintegral
        $$
        \int_\alpha z e^{z^2} d z
        $$
    \end{problem}

    \begin{proof}
        Die Kurve entspricht einem "8er", also zuerst ein Kreis um $(1,0)$ der mit positiver Orientierung durchlaufen
        wird und dann ein Kreis um $(-1,0)$ der mit negativer Orientierung durchlaufen wird. \\\\
        Da $ze^{z^2}$ und $\alpha(0) = 1-e ^{i0} = 0 = -1+e^{-i 4\pi} = \alpha(4\pi)$ ist die Kurve geschlossen
        und die Funktion holomorph. Damit ist das gegebene Integral $0$.\\\\
        Oder Stammfunktion raten.
    \end{proof}

    \begin{problem}{3}
        Seien $\Omega \subset \mathbb{C}$ offen und $f: \Omega \rightarrow \mathbb{C}$ holomorph. Sei $R \subset \mathbb{C}$ ein solides Rechteck
        und $\varphi: \mathbb{C} \rightarrow \mathbb{C}$ eine bijektive $C^1$ Funktion mit $\varphi(R) \subset \Omega$. Zeige: Parametrisiert $\gamma$ den Rand
        von $\varphi(R)$, so gilt
        $$
        \int_\gamma f(z) d z=0 .
        $$
    \end{problem}

    \begin{proof}
        Wir adaptieren den Beweis von Goursat für Rechtecke. Dafür unterteilen wir unser Rechteck
        $R^{(0)}$ mit positiver Orientierung und $d^{(0)}$ und $u^{(0)}$ für Durchmesser und Umfang.
        Wir verbinden nun gegenüberliegende Mittelpunkte um so $4$ neue Rechtecke zu konstruieren.
        Eines dieser Rechtecke $R^{(1)}$ erfüllt
        $$\left|\int_{R^{(0)}} f(z)dz\right|=\left|\int_{R^{(1)}_1} f(z)dz+\int_{R^{(1)}_2} f(z)dz+
        \int_{R^{(1)}_3} f(z)dz+\int_{R^{(1)}_4} f(z)dz\right|\leq 4 \left|\int_{R^{(1)}} f(z)dz\right|$$
        Weiter gilt $d^{(1)} = \frac{1}{2}d^{(0)}$ und $p^{(1)}=\frac{1}{2}p^{(0)}$.\\\\
        Nach Proposition 1.4 finden wir einen eindeutigen Punkt $z_0$ der in $\lim_{n\to\infty}R^{(n)}$ liegt.
        Mit $f$ holomorph finden wir somit $f(z) = f(z_0)+f'(z_0)(z-z_0)+\psi(z)(z-z_0)$. $f(z_0)$ sowie
        $f'(z_0)(z-z_0)$ haben als konstante bzw. lineare Funktion Stammfunkionen, also gilt
        $$\int_{R^{(n)}} f(z)dz = \int_{R^{(n)}}\psi(z)(z-z_0)dz$$
        Nun muss natürlich im Rechteck $R^{(n)}$ $|z-z_0|\leq d^{(n)}$ gelten. Somit haben wir:
        $$\left|\int_{R^{(n)}} f(z)dz\right|\leq \sup_{z\in R^{(n)}}|\psi(z)| d^{(n)}p^{(n)} \leq \varepsilon_n 4^{-n}
        d^{(0)}p^{(0)}$$
        Gesamt gilt damit:
        $$\left|\int_{R^{(0)}} f(z)dz\right|\leq 4^n\left|\int_{R^{(n)}} f(z)dz\right|\leq \varepsilon_n d^{(0)}p^{(0)}$$
        Für $n\to\infty$ geht $\varepsilon_n\to 0$.\\\\
        $\varphi$ bijektiv, stetig, kompakt, also Homoömorphismus. Also bildet
        der Rand von $R$ auf den Rand von $\varphi(R)$. \\
        $\varphi$ außerdem Lipschitzstetig, weil wir auf einem kompakten Intervall
        arbeiten. Dadurch kann man Umfang und Durchmesser mit $L$ abschätzen. \\
        Dann Beweis von Goursat, jeweils mit $L$.
    \end{proof}

    \begin{problem}{4}
        Berechne die folgenden Integrale:
        \begin{enumerate}[label = (\alph*)]
            \item Für $n \in \mathbb{Z}$ :
            $$
            \int_C z^n d z
            $$
            wobei $C$ ein im Ursprung zentrierter, positiv orientierter Kreis ist.
            \item Für $n \in \mathbb{Z}$ :
            $$
            \int_C z^n d z
            $$
            wobei $C$ der positiv orientierte Rand einer Kreisscheibe $\partial B_R(z)$ ist mit $0<R<|z|$.
            \item Für $0<a<r<b$ :
            $$
            \int_C \frac{1}{(z-a)(z-b)} d z,
            $$
            wobei $C=\{z \in \mathbb{C}:|z|=r\}$ mit positiver Orientierung.
        \end{enumerate}
    \end{problem}

    \begin{proof}
        \begin{enumerate}[label = (\alph*)]
            \item Weil $C$ geschlossen und $z^n$ holomorph für $n\geq 0$ ist, verschwindet das Integral in diesen
            Fällen. Wenn $n< 0$ finden wir mit der Parametrisierung $\varphi: [0,2\pi]\to\C, t\mapsto re^{it}$
            $$\int_C z^n d z = \int_{0}^{2\pi} (re^{it})^n rie^{it}\,dt = ir^{n+1}\int_{0}^{2\pi} e^{it(n+1)}\,dt
            $$
            Im Fall $n=-1$ ergibt diese Integral $i2\pi$, andernfalls erhalten wir $\left[\frac{ir^{n+1}e^{it(n+1)}}
            {i(n+1)}\right]_0^{2\pi}=0$
            \item Alle gegebenenen Funktionen sind Zusammensetzungen holomorpherfunktionen (deren Quotient nirgendwo
            verschwindet), also ist das Integral jeweils $0$. Für $n\neq -1$ finden wir die Stammfunktion $F(z) = \frac{z^{n+1}}{n+1}$, da der Kreis
            eine geschlossene Kurve ist, finden wir (mit einer beliebigen Parametrisierung von $z_0$ nach
            $z_0$):
            $$\int_C z^n d z = F(z_0)-F(z_0) = 0$$
            Nun behandeln wir den Fall $n=-1$. Dafür nehmen wir die Kurve $\varphi: [0,2\pi)\to \C, t\mapsto
            z + R\cdot e^{it}$. Damit finden wir:
            $$\int_C z^{-1} d z = \int_{0}^{2\pi} \frac{1}{z + R\cdot e^{it}}\cdot Rie^{it}\,dt =
            \int_{z+R}^{z+R}\frac{1}{u}\, du = 0$$
            \item Wir parametrisieren die Kurve mit $\varphi: [0,2\pi]\to\C, t\mapsto re^{it}$ und finden damit:
            $$\begin{aligned}
                  \int_C \frac{1}{(z-a)(z-b)} d z &=\int_C -\frac{1}{(b-a)(z-a)}dz + \int_C \frac{1}{(b-a)(z-b)}dz
                  \\&= \frac{1}{a-b}\left(\int_C \frac{1}{z-a}\,dz+\int_C \frac{1}{b-z}\,dz\right)
                  \\&= \frac{1}{a-b}\left(\int_C \frac{1}{z}\frac{1}{1-\frac{a}{z}}\,dz+
                  \int_C \frac{1}{b}\frac{1}{1-\frac{z}{b}}\,dz\right)
                  \\&= \frac{1}{a-b}\left(\int_C \frac{1}{z}\sum_{n=0}^\infty \left(\frac{a}{z}\right)^n\,dz+
                  \int_C \frac{1}{b}\sum_{n=0}^\infty \left(\frac{z}{b}\right)^n\,dz\right)
                  \\&= \frac{1}{a-b}\left(\sum_{n=0}^\infty a^n \int_C \frac{1}{z^{n+1}}\,dz+
                  \sum_{n=0}^\infty\frac{1}{b^{n+1}}\int_C z^n\,dz\right)
                  \\&= \frac{1}{a-b}\left(\sum_{n=0}^\infty a^n \int_0^{2\pi} \frac{1}{(re^{i\pi})^{n+1}} rie^{i\pi}\,dt+
                  \sum_{n=0}^\infty\frac{1}{b^{n+1}}\int_0^{2\pi} (re^{i\pi})^{n+1} rie^{i\pi}\,dt\right)
                  \\&= \frac{i}{a-b}\left(\sum_{n=0}^\infty \left(\frac{a}{r}\right)^n \int_0^{2\pi}
                  e^{-itn} \,dt+\sum_{n=0}^\infty\left(\frac{r}{b}\right)^{n+1}\int_0^{2\pi}e^{it(n+1)} \,dt\right)
                  \\&= \frac{i}{a-b}\left(2\pi+ \sum_{n=1}^\infty \left(\frac{a}{r}\right)^n \frac{-1}{in}
                  [e^{-itn}]_0^{2\pi}+\sum_{n=0}^\infty\left(\frac{r}{b}\right)^{n+1}\frac{1}{i(n+1)}
                  [e^{-it(n+1)}]_0^{2\pi}\right)
                  \\&= \frac{2\pi i}{a-b}
            \end{aligned}$$
            Funktioniert auch mit dem Representationssatz.
        \end{enumerate}
    \end{proof}

    \begin{problem}{5}
        Sei $f$ holomorph auf $\Omega \subset \mathbb{C}$, und $T \subset \Omega$ ein Dreieck dessen
        Inneres auch in $\Omega$ enthalten ist. Gemäss Satz von Goursat gilt dann, dass
        $$
        \int_T f(z) d z=0 .
        $$
        Beweise dies mittels Satz von Green unter der zusätzlichen Annahme, dass die Ableitung $f^{\prime}$
        stetig ist.
    \end{problem}

    \begin{proof}
        Der Satz von Green besagt:
        $$\int_T F\,dx +G\,dy = \int_{\text{Inneres von }T}
        \left(\frac{\partial G}{\partial x}-\frac{\partial F}{\partial y}\right)\,dxdy$$
        Wir bezeichnen im Folgenden das Innere des Dreiecks mit $T^\circ$\\
        Wir schreiben nun $f(z) = u(x,y)+i v(x,y)$ sowie $dz = dx + idy$:
        $$\int_T f(z) d z = \int_T (u+iv)(dx+idy) = \int_T u\,dx - v\,dy + i \int_T u\,dy+v\,dx$$
        Mit dem Satz von Green:
        $$ = \int\hspace{-5pt}\int_{T^\circ} \left(-\frac{\partial v}{\partial x}-
        \frac{\partial u}{\partial y}\right)\,dxdy+i\int\hspace{-5pt}\int_{T^\circ} \left(\frac{\partial v}
        {\partial y}-\frac{\partial u}{\partial x}\right)\,dxdy = 0$$
        Wobei die letzte Gleichheit aufgrund der Cauchy-Riemann-Gleichungen gilt.
    \end{proof}
\end{document}