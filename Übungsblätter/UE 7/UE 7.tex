\documentclass[11pt]{article}
\title{Zusammenfassung Numerik 2022 WS}
\author{Simon Garger}
\usepackage{amsmath}
\usepackage{amssymb}
\usepackage{amsfonts}
\usepackage{setspace}
\usepackage{fullpage}
\usepackage{blkarray}
\usepackage[a4paper,left=2.5cm,right=2.5cm,top=2cm,bottom=4cm,bindingoffset=5mm]{geometry}
\usepackage{graphicx}
\usepackage{wasysym}
\usepackage{mathtools}
\usepackage{titlesec}
\usepackage{xcolor}
\usepackage{bm}
\usepackage{enumitem}
\usepackage{amsbooka}
\usepackage{translit}
\usepackage{arabicore}
\usepackage{amsmath-2018-12-01}
\usepackage{amsmath-2018-12-01}
\usepackage{farsifnt}


\newcommand{\N}{\mathbb{N}}
\newcommand{\Z}{\mathbb{Z}}
\newcommand{\R}{\mathbb{R}}
\newcommand{\C}{\mathbb{C}}
\newcommand{\res}{\text{res}}
\renewcommand{\Re}{\mathfrak{Re}}
\renewcommand{\Im}{\mathfrak{Im}}


\newenvironment{problem}[2][Beispiel]{
    \begin{trivlist}
        \item[\hskip \labelsep {\bfseries #1}\hskip \labelsep {\bfseries #2.}] \itshape}{
    \end{trivlist}\normalshape
}

\begin{document}
    \begin{problem}{1}
        Zeige: Falls $h: B_1(0) \rightarrow \mathbb{R}$ harmonisch ($\Delta h(x, y)=\frac{\partial^2 h}{\partial x^2}
        +\frac{\partial^2 h}{\partial y^2}=0$ für $z=x+i y \in B_1(0)$) auf der Einheitskreisscheibe,
        dann existiert eine holomorphe Funktion $f: B_1(0) \rightarrow \mathbb{C}$ so dass
        $\operatorname{Re}(f(z))=h(x, y)$.
        (Hinweis: Konstruiere zuerst die Ableitung von $f$.)
        Gilt dies auch falls $B_1(0)$ durch $B_1(0) \backslash\{0\}$ ersetzt wird?
    \end{problem}

    \begin{proof}
        Wir betrachten zuerst die Funktion:
        $$g(z):= \left(\frac{\partial h}{\partial x} - i\frac{\partial h}{\partial y}\right)(z)$$
        Diese Funktion ist holomorph, denn sie erfüllt die Cauchy-Riemann-Gleichungen:
        $$\frac{\partial }{\partial x}\frac{\partial h}{\partial x}=\frac{\partial^2 h}{\partial^2 x}
        =-\frac{\partial^2 h}{\partial^2 y}=\frac{\partial}{\partial y}\left(-\frac{\partial h}{\partial y}
        \right)\qquad\qquad \frac{\partial }{\partial y}\frac{\partial h}{\partial x} =
        -\frac{\partial }{\partial x}\left(-\frac{\partial h}{\partial y}\right)$$
        Nun betrachten wir die Funktion:
        $$f(z) := h(z_0) + \int_\gamma g(\xi)\,d\xi$$
        wobei $\gamma$ ein Weg von $z_0$ bis $z$ ist. Weil auf dem Einheitskreis alle Wege
        homotop sind, ist diese Definition eindeutig.  Diese Funktion ist wieder holomorph, weil sie differenzierbar ist
        sei nun $u(z)$ der Realteil von $f$. Dann gilt zuerst einmal $u(z_0)=h(z_0)$ und weiter gilt:
        $$\left(\frac{\partial h}{\partial x} - i\frac{\partial h}{\partial y}\right)(z)=f'(z) =
        \left(\frac{\partial u}{\partial x}-i\frac{\partial u}{\partial y}\right)(z)$$
        Wobei die letzte Gleichheit aus Cauchy-Riemann und $\frac{\partial}{\partial z}=\frac{1}{2}\left(
        \frac{\partial}{\partial x}-i\frac{\partial}{\partial y}\right)$ folgt.\\
        Damit sind die partiellen Ableitungen von $h$ und $u$ gleich und sie stimmen an $z_0$ überein, also sind sie
        insgesamt gleich.
        \\\\
        Das gilt nur für einfachzusammenhängende Gebiete. Zum Beispiel ist $h(x,y)=\ln(\sqrt{x^2+y^2})
        =\Re(\log(z))$ wobei $\log(z)$ in diesem Kontext für den Hauptzweig des Logarithmus steht.
        Als Realteil einer auf der geschlitzen Einheitskreisscheibe ist die Funktion harmonisch.
        Das kann man auch über die partiellen Ableitungen argumentieren, denn es gilt:
        $$\frac{\partial^2 h}{\partial x^2} =\frac{\partial h}{\partial x} \frac{x}{x^2+y^2} =
        \frac{-x^2+y^2}{(x^2+y^2)^2}$$
        Da die Funktion symmetrisch in $x$ und $y$ ist folgt damit, dass die Funktion harmonisch ist.\\\\
        Nehmen wir nun an es gäbe eine holomorphe Funktion mit $f(z) = h(z) + iv(z)$.
        Nach Cauchy-Riemann gilt dann (mit den oben berechneten partiellen Ableitungen):
        $$f'(x,y) = \left(\frac{\partial h}{\partial x}-i\frac{\partial h}{\partial y}\right)(x,y) =
        \frac{x}{x^2+y^2}-i\frac{y}{x^2+y^2} = \frac{x-iy}{x^2+y^2}=\frac{\bar{z}}{|z|^2} = \frac{1}{z}$$
        Nach dem Kommentar unter Korollar 3.3 in den Princeton Lectures Complex Analysis (S. 23) wissen
        wir, dass $\frac{1}{z}$ auf der durchstochenen Einheitskreisscheibe keine Stammfunktion
        hat. Also kann diese Funktion $f$ nicht auf der durchstochenen Einheitskreisscheibe
        definiert sein.
    \end{proof}

    \begin{problem}{2}
        Sei $\Omega \subset \mathbb{C}$ ein Gebiet. Zeige dass falls eine der folgenden Bedingungen
        erfüllt ist, $\Omega$ einfach zusammenhängend ist:
        \begin{enumerate}[label = (\alph*)]
            \item $\Omega$ ist konvex (d.h. für je zwei Punkte in $\Omega$ liegt auch das sie verbindende
        Geradensegment in $\Omega$ ).
            \item Es gibt einen Punkt $z_0 \in \Omega$ so dass für alle $z \in \Omega$ das Geradensegment zwischen
        $z$ und $z_0$ in $\Omega$ liegt. (Ein solches $\Omega$ heisst sternförmig.)
        \end{enumerate}
        Zeige: die geschlitzte Ebene $\mathbb{C} \backslash(-\infty, 0]$ ist einfach zusammenhängend.
    \end{problem}

    \begin{proof}
        Seien im folgenden $\gamma_1$ und $\gamma_2$ beliebige Wege in $\Omega$ mit $\gamma_1(0)=\gamma_2(0)=
        \alpha$ und $\gamma_1(1)=\gamma_2(1)=\beta$. Sowohl in a) als auch in b) finden wir solche Wege, weil
        beide Mengen wegzusammenhängend sind.
        \begin{enumerate}[label = (\alph*)]
            \item Wir betrachten die Homotopie:
            $$F(s,t) = s\gamma_2(t)+(1-s)\gamma_1(t)$$
            Das ist eine stetige Funktion, die jeweils für alle $t$ die Kurven Segmente $\gamma_1(t)$ und
            $\gamma_2(t)$ interpoliert. Diese "Zwischenkurven" liegen jeweils in der Menge, weil wir für jedes
            $t$ die Konvexität der Menge nutzen. Weiter gilt $F(s,0)=\alpha$ und $F(s,1)=\beta$ sowie
            $F(0,t)=\gamma_1(t)$ und $F(1,t)=\gamma_2(t)$. Also erfüllen die Funktionen genau unsere Definition
            von homotop und somit ist $\Omega$ einfach zusammenhängend.
            \item Sei
            $$F(s,t):=
            \begin{cases}
                (1-3t)\alpha + 3t((1-s)\gamma_1(0)+sz_0)\qquad & t\in\left[0,\frac{1}{3}\right]\\
                (1-s)\gamma_1(3t-1)+sz_0\qquad & t\in\left(\frac{1}{3},\frac{2}{3}\right)\\
                (3-3t)((1-s)\gamma_1(1)+sz_0) + (3t-2)\beta\qquad & t\in\left[\frac{2}{3},1\right]
            \end{cases}$$
            Das ist genau die Kurve, die die ursprüngliche Kurve immer weiter zu $z_0$ zieht, die Anfangs- und
            Endpunkt dabei aber konstant lässt und linear mit den zusammengezogenen Punkten verbindet.
            Per Konstruktion ist die Funktion in $t$ stetig. In $s$ ist die
            Funktion stetig, weil $s$ nur als Kontraktionsfaktor vorkommt. Somit ist jede Kurve zu der Funktion,
            die von $\alpha$ bis $z_0$ eine Gerade ist und von $z_0$ bis $\beta$ homotop. Damit sind aber auch
            $\gamma_1$ und $\gamma_2$ homotop, weil die Eigenschaft transitiv ist.
        \end{enumerate}
        Die geschlitzte Ebene ist einfach zusammenhängend, weil sie sternförmig ist. Denn wir betrachten
        $z_0=1$. Für alle Zahlen auf der positiven $x$-Achse liegt die Verbindungslinie offensichtlich in
        $\Omega = \mathbb{C} \backslash(-\infty, 0]$ für alle anderen $z$ ist der Imaginärteil ungleich
        $0$ und somit auch entlang der Verbindungslinie ungleich $0$, also in $\Omega$.
    \end{proof}

    \begin{problem}{3}
        Berechne fïr $a>0$ das Integral
        $$
        \int_0^{\infty} \frac{\log (x)}{x^2+a^2} d x
        $$
        (Hinweis: Wähle einen passenden Zweig des Logarithmus und eine Kontour, welche den Ursprung meidet.)
    \end{problem}

    \begin{proof}
        Wir betrachten die Funktion $f(z)=\frac{\log(z)}{z^2+a^2}$, also $\theta\in[-\pi/2,3\pi/2)$.
        Wir wählen den Logarithmus mit dem Schlitz entlang der negativen imaginären Achse. Wir werden
        entlang eines Halbkreises mit Radius $R$ auf der oberen Halbebene integrieren und mit einem kleinen
        Halbkreis mit Radius $r$ den Ursprung ausnehmen. Darin befindet sich das Residuum $ia$. Dafür finden
        wir:
        $$\res_{ia} f(z) = \lim_{z\to ia}\frac{\log(z)}{z+ia}=\frac{\log(ia)}{2ia} = \frac{\log(a)+i\frac{\pi}{2}}
        {2ia}$$
        Damit haben wir nach dem Residuensatz:
        $$\int_r^R \frac{\log (x)}{x^2+a^2} d x+ \int_{-R}^{-r}\frac{\log(x)}{x^2+a^2} d x +
        \int_{\gamma_R}f(z)\,dz+\int_{\gamma_r}f(z)\,dz = \pi\frac{\log(a)+i\frac{\pi}{2}}{a}$$
        Nun berechnen wir die einzelnen Integrale:
        $$\begin{aligned}
              &\int_{-R}^{-r} \frac{\log(x)}{x^2+a^2} d x = \int_{-R}^{-r} \frac{\log|x|+i\pi}{x^2+a^2} d x
              = \int_{-R}^{-r}  \frac{\log|x|}{x^2+a^2} d x+\int_{-R}^{-r} \frac{i\pi}{x^2+a^2} d x =\\
              &\int_{r}^{R}\frac{\log|x|}{x^2+a^2} d x + \frac{i\pi}{a}\arctan\left(\frac{-r}{a}\right)
              -\frac{i\pi}{a}\arctan\left(\frac{-R}{a}\right)\to \int_0^\infty \frac{\log (x)}{x^2+a^2} d x +
              \frac{i\pi^2}{2a}
        \end{aligned}$$
        Weiter sehen wir:
        $$\begin{aligned}
              &\left|\int_{\gamma_R}f(z)\,dz\right|=\left|\int_0^{\pi}\frac{\log(Re^{it})iRe^{it}}{R^2e^{2it}+a^2}dt
              \right|\leq \int_0^{\pi}\frac{|\log(R)+it|R}{R^2-a^2}dt\\&\leq
              \int_0^{\pi}\frac{(\log(R)+\pi)R}{R^2-a^2}dt = \frac{\pi(\log(R)+\pi)R}{R^2-a^2}\to 0
        \end{aligned}$$
        Für das Integral über den kleinen Halbkreis gilt:
        $$\left|\int_{\gamma_r}f(z)\,dz\right| = \left|\int_0^\pi\frac{\log(re^{it})ire^{it}}{r^2e^{2it}+a^2}
        dt\right|\leq \int_0^\pi\frac{|\log(r)+\pi|r}{a^2-r^2}dt\leq \frac{\pi (\log(r)+\pi)r}{a^2-r^2}\to 0$$
        Gesamt haben wir also:
        $$\pi\frac{\log(a)+i\frac{\pi}{2}}{a} = 2 \int_0^\infty \frac{\log (x)}{x^2+a^2} d x +
        \frac{i\pi^2}{2a}\Rightarrow \int_0^\infty \frac{\log (x)}{x^2+a^2} d x = \frac{\pi\log(a)}{2a}$$
    \end{proof}

    \begin{problem}{4}
        \begin{enumerate}[label = (\alph*)]
            \item Sei $f: \mathbb{R} \rightarrow \mathbb{C}$ stetig und $2 \pi$-periodisch (d.h., $f(x+2 \pi)=f(x)
            , x \in \mathbb{R}$ ). Zeigen Sie folgendes:
            $$
            \int_{-\pi}^\pi f(x) d x=\int_{-\pi}^\pi f(x+a) d x=\int_{-\pi+a}^{\pi+a} f(x) d x, \quad a \in
            \mathbb{R} .
            $$
            \item Angenommen, $\left\{a_n\right\}_{n-1}^N$ und $\left\{b_n\right\}_{n=1}^N$ seien zwei endliche
            Folgen komplexer Zahlen. Sei
            $$
            B_n:=\sum_{k=1}^n b_k, \quad B_0=0
            $$
            die Reihe der Teilsummen. Beweisen Sie die Formel für die Summierung nach Teilen
            $$
            \sum_{n=M}^N a_n b_n=a_N B_N-a_M B_{M-1}-\sum_{n=M}^{N-1}\left(a_{n+1}-a_n\right) B_n .
            $$
        \end{enumerate}
    \end{problem}

    \begin{proof}
        \begin{enumerate}[label = (\alph*)]
            \item Aus der Substitution $u=x+a$ folgt die zweite Gleichheit mit:
            $$\int_{-\pi}^\pi f(x+a) d x=\int_{-\pi+a}^{\pi+a}f(u)du$$
            Weiter sehen wir:
            $$\frac{d}{dx}\int_{x}^{x+2\pi}f(t)dt = f(x+2\pi)-f(x)=f(x)-f(x)=0$$
            Damit folgt, dass es egal ist über welches Intervall wir integrieren, wenn es Länge $2\pi$ hat.
            Damit ist der rechte und ganz linke Ausdruck gleich.
            \item Wir beweisen per Induktion nach $N$ mit $(M\leq N)$ für $N=1$ sehen wir:
            $$\sum_{n=1}^1 a_n b_n=a_1b_1-\underbrace{a_1 B_{0}-\sum_{n=1}^{0}\left(a_{n+1}-a_n\right) B_n}
            _{=0}$$
            Sei das Ganze für $N$ und beliebiges $M\leq N$ gezeigt, dann sehen wir:
            $$\begin{aligned}
                  \sum_{n=M}^{N+1} a_n b_n
                  &= a_{N+1}b_{N+1} + a_N B_N-a_M B_{M-1}-\sum_{n=M}^{N-1}\left(a_{n+1}-a_n\right) B_n\\
                  &= \underbrace{a_{N+1}b_{N+1}+a_{N+1}B_N}_{=a_{N+1}B_{N+1}}-\underbrace{a_{N+1}B_N +
                  a_N B_N}_{(a_{N+1}-a_N)B_N}-a_M B_{M-1}-\sum_{n=M}^{N-1}\left(a_{n+1}-a_n\right) B_n\\
                  &= a_{N+1} B_{N+1}-a_M B_{M-1}-\sum_{n=M}^{N}\left(a_{n+1}-a_n\right) B_n
            \end{aligned}$$
            Für $M=N$ gilt:
            $$\sum_{n=N}^N a_n b_n=a_Nb_N = a_N B_N-a_N B_{N-1}-\underbrace{\sum_{n=N}^{N-1}\left(a_{n+1}-a_n\right)
            B_n}_{=0}$$
        \end{enumerate}
    \end{proof}

    \begin{problem}{5}
        \begin{enumerate}[label = (\alph*)]
            \item Sei $\left\{a_n: n \geq 0\right\} \subset \mathbb{C}$ eine Folge von komplexen Zahlen und
            $$
            b_n:=\frac{1}{n} \sum_{k=0}^{n-1} a_n, \quad n \geq 1 .
            $$

            Zeigen Sie, dass wenn $a_n \rightarrow a \in \mathbb{R}$, dann $b_n \rightarrow a$, für $n
            \rightarrow \infty$.
            \item Eine (formelle) Reihe von komplexen Zahlen $\sum_{n \geq 1} a_n$ heißt Cesàro konvergent,
            wenn die Folge der Durchschnittswerte der Teilsummen
            $$
            C_n:=\frac{1}{n} \sum_{k=0}^{n-1} \sum_{j=1}^k a_j, \quad n \geq 1,
            $$
            konvergiert (mit Cesàro Limes $=\lim _n C_n$ ). Zeigen Sie, dass eine konvergente Reihe auch Cesàro
            konvergent ist.
            \item Zeigen Sie, dass die Reihe $\sum_{n \geq 1}(-1)^n$ nicht konvergent ist, aber sie Cesàro
            konvergent ist. Was ist ihr Cesàro Limes?
            \item Sei $\sum_{\mathrm{n} \geq 1} a_n$ eine Cesàro konvergente Reihe von komplexen Zahlen.
            Zeigen Sie, dass $\frac{a_n}{n} \rightarrow 0$ für $n \rightarrow \infty$.
        \end{enumerate}
    \end{problem}

    \begin{proof}
        \begin{enumerate}[label = (\alph*)]
            \item Sei $\varepsilon>0$ beliebig, dann finden wir $N$ mit $|a_n-a|<\varepsilon$ für $n\geq N$.
            Somit haben wir:
            $$\begin{aligned}
                  |b_{n}-a|&= \left|\frac{1}{n}\sum_{k=0}^{N-1}a_k+ \frac{1}{n}\sum_{k=N}^{n} a_k-a\right|
                  \leq \frac{|\sum_{k=0}^{N-1}a_k|}{n} + \frac{\sum_{k=N}^{n}|a_k-a|}{n}\\&\leq
                  \frac{|\sum_{k=0}^{N-1}a_k|}{n} + \frac{(n-N+1)\varepsilon}{n}\to\varepsilon
            \end{aligned}$$
            \item Die Aussage folgt aus (a) und der konvergenten Folge $\tilde{a}_k =\sum_{j=1}^k a_j$.
            \item Offensichtlich ist die Reihe nicht konvergent, denn wir finden kein $N$ sodass die
            Folgenglieder in einem Ball mit Radius $\frac{1}{3}$ liegen. Allerdings sehen wir:
            $$\frac{1}{n}\sum_{k=0}^{n-1} \sum_{j=1}^k (-1)^j=\frac{1}{n}\sum_{k=1}^{\lceil\frac{n-1}{2}\rceil}(-1)
            =\frac{1-\lceil\frac{n-1}{2}\rceil}{n}$$
            Für $n$ ungerade haben wir:
            $$\frac{1-\lceil\frac{n-1}{2}\rceil}{n}=\frac{1-\frac{n-1}{2}}{n}\to-\frac{1}{2}$$
            und für $n$ gerade:
            $$\frac{1-\lceil\frac{n-1}{2}\rceil}{n}=\frac{1-\frac{n}{2}}{n}\to-\frac{1}{2}$$
            In beiden Fällen hat die Folge einen Grenzwert. Dieser stimmt überein, also ist der Grenzwert
            $-\frac{1}{2}$.
            \item Wir bezeichnen $S_m = \sum_{k=1}^m a_k$ und $C_n =\frac{1}{n}\sum_{k=1}^{n-1}S_k$.
            Nach Voraussetzung gilt $C_n-C_{n-1}\to C-C=0$. Damit gilt:
            $$\begin{aligned}
                  C_n-C_{n-1} &= \frac{1}{n}\sum_{k=1}^{n-1}S_k - \frac{1}{n-1}\sum_{k=1}^{n-2}S_k\\
                  &=\frac{1}{n}\sum_{k=1}^{n-1}S_k-\frac{1}{n}\sum_{k=1}^{n-2}S_k
                  + \frac{1}{n}\sum_{k=1}^{n-2}S_k- \frac{1}{n-1}\sum_{k=1}^{n-2}S_k \\
                  &= \frac{1}{n}S_{n-1} - \frac{1}{n}\frac{1}{n-1}\sum_{k=1}^{n-2}S_k\\
                  &= \frac{1}{n}S_n - \frac{1}{n}C_{n-1}\to \frac{1}{n}S_n
            \end{aligned}$$
            Nun wissen wir, dass $\frac{1}{n}S_n$ auch gegen $0$ gehen muss. Es gilt aber auch:
            $$\frac{S_{n-1}}{n} = \frac{S_{n-1}}{n-1}\cdot\frac{n-1}{n}\to 0\cdot 1$$
            Also gilt auch:
            $$\frac{a_n}{n}=\frac{S_n-S_{n-1}}{n}\to 0-0=0$$
        \end{enumerate}
    \end{proof}
\end{document}