\documentclass[11pt]{article}
\title{Zusammenfassung Numerik 2022 WS}
\author{Simon Garger}
\usepackage{amsmath}
\usepackage{amssymb}
\usepackage{amsfonts}
\usepackage{setspace}
\usepackage{fullpage}
\usepackage{blkarray}
\usepackage[a4paper,left=2.5cm,right=2.5cm,top=2cm,bottom=4cm,bindingoffset=5mm]{geometry}
\usepackage{graphicx}
\usepackage{wasysym}
\usepackage{mathtools}
\usepackage{titlesec}
\usepackage{xcolor}
\usepackage{bm}
\usepackage{enumitem}
\usepackage{amsbooka}
\usepackage{translit}
\usepackage{arabicore}
\usepackage{amsmath-2018-12-01}
\usepackage{amsmath-2018-12-01}
\usepackage{farsifnt}


\newcommand{\N}{\mathbb{N}}
\newcommand{\Z}{\mathbb{Z}}
\newcommand{\R}{\mathbb{R}}
\newcommand{\C}{\mathbb{C}}
\newcommand{\res}{\text{res}}
\renewcommand{\Re}{\mathfrak{Re}}
\renewcommand{\Im}{\mathfrak{Im}}


\newenvironment{problem}[2][Beispiel]{
    \begin{trivlist}
        \item[\hskip \labelsep {\bfseries #1}\hskip \labelsep {\bfseries #2.}] \itshape}{
    \end{trivlist}\normalshape
}

\begin{document}
    Satz von Parseval: Seien $f \in L^2([-\pi, \pi])$ und
    $$
    S_N(f)(x):=\sum_{n=-N}^N \hat{f}(n) e^{i n x}, \quad N \geq 1
    $$
    die partielle Fourier-Summen. Dann gilt Folgendes.
    \begin{itemize}
        \item (Quadratische Konvergenz)
        $$
        \frac{1}{2 \pi} \int_{-\pi}^\pi\left|f(x)-S_N(f)(x)\right|^2 d x \longrightarrow 0 \text { für } N \rightarrow \infty .
        $$
        \item (Parseval-Isometrie)
        $$
        \frac{1}{2 \pi} \int_{-\pi}^\pi|f(x)|^2 d x=\sum_{n \in \mathbb{Z}}|\hat{f}(n)|^2 .
        $$
    \end{itemize}

    \begin{problem}{1}
        Sei $f:[0, \pi] \rightarrow \mathbb{R}$ stetig mit $f(0)=f(\pi)=0$. Zeigen Sie, dass eine
        stetige Funktion $u:[0, \pi] \times[0, \infty)$ existiert, so dass
        $$
        \begin{cases}u(0, t)=0, & t \in[0, \infty) \\ u(\pi, t)=0, & t \in[0, \infty) \\ u(x, 0)=f(x),
        & x \in[0, \pi] \\ \Delta u=0 & \text { auf }(0, \pi) \times(0, \infty)\end{cases}
        $$

        Hinweis: Betrachten Sie $g:[-\pi, \pi] \rightarrow \mathbb{R}$, mit $g(-x)=-g(x), g \equiv f$
        auf $[0, \pi]$ und die Funktionen $u_k(x, t)=e^{-k t} \sin (k x)$.
    \end{problem}

    \begin{proof}
        Zuerst sehen wir durch
        $$\Delta u_k(x,t) = \underbrace{e^{-kt}(-k^2\sin(kx))}_{\frac{\partial^2 u_k}{\partial x^2}}
        + \underbrace{k^2e^{-kt}\sin(kx)}_{\frac{\partial^2 u_k}{\partial y^2}}=0$$
        , dass die Funktionen tatsächlich harmonisch sind (und damit auch alle linear Kombinationen).
        Weiter erinnern wir uns, dass für alle ungeraden Funktionen $g$
        $$g(x)\sim 2i\sum_{n\geq 1}\hat{g}(n)\sin(nx)$$
        gilt. Wir definieren mit diesem Wissen:
        $$u(x,t):=2i\sum_{n\geq 1}\hat{g}(n) u_n(x,t)=2i\sum_{n\geq 1}\hat{g}(n) e^{-nt}\sin(nx)$$
        Nun prüfen wir die gewünschten Eigenschaften:
        \begin{enumerate}[label = (\roman{enumi})]
            \item $\sin(0)=0\Rightarrow u(0,t)=0$
            \item $\sin(\pi)=0\Rightarrow u(\pi,t)=0$
            \item $e^{0}=1\Rightarrow g(x)\sim u(x,0)$
            \item $\forall k|\Delta u_k=0\Rightarrow \Delta u=0$
        \end{enumerate}
        Bleibt zu zeigen, dass die Fourierreihe von $g$ gegen die Funktion selbst konvergiert.
        Das ist der Fall, denn wenn wir $r:=e^{-t}$ dann haben wir für positives $r$ genau das
        $u(x,t)$ die Abelreihe von $g$ ist und somit konvergiert $u$ gegen $g$ für $t\to 0$.
    \end{proof}

    \begin{problem}{2}
    (a) Sei $f$ die Funktion, die auf $[-\pi, \pi]$ durch $f(x)=|x|$ definiert ist. Verwenden Sie
    die Parsevals Identität, um das Folgende zu zeigen:
    $$
    \sum_{n=0}^{\infty} \frac{1}{(2 n+1)^4}=\frac{\pi^4}{96}, \quad \sum_{n=1}^{\infty} \frac{1}{n^4}
    =\frac{\pi^4}{90} .
    $$
    (b) Betrachten Sie die $2 \pi$-periodische ungerade Funktion, die auf $[0, \pi]$ durch
    $f(x)=$ $x(\pi-x)$ definiert ist. Zeigen Sie,
    $$
    \sum_{n=0}^{\infty} \frac{1}{(2 n+1)^6}=\frac{\pi^6}{960}, \quad \sum_{n=1}^{\infty} \frac{1}{n^6}
    =\frac{\pi^6}{945} .
    $$
    \end{problem}

    \begin{proof}
        Wir rechnen mit Parseval und finden dafür zuerst:
        $$\frac{1}{2\pi}\int_{-\pi}^{\pi} |f|^2dx=\frac{1}{2\pi}\int_{-\pi}^{\pi} x^2dx
        =\frac{1}{\pi}\int_{0}^{\pi} x^2dx=\frac{1}{3\pi}\pi^3=\frac{\pi^2}{3}$$
        Für $n\neq 0$ finden wir mit der bereits aus anderen Aufgaben bekannten Stammfunktion
        $\frac{e^{-inx}}{n^2}-\frac{x}{in}e^{-inx}$ von $xe^{-inx}$:
        $$\begin{aligned}
              \hat{f}(n) &= \frac{1}{2\pi}\int_{-\pi}^{\pi} |x|e^{-inx}dx =
              \frac{1}{2\pi}\left(-\int_{-\pi}^{0} xe^{-inx}dx+\int_{0}^{\pi} xe^{-inx}dx\right)\\
              &= \frac{1}{2\pi}\left(-\frac{1}{n^2}+\frac{(-1)^n}{n^2}+\frac{(-1)^n\pi}{in}+
              \frac{(-1)^n}{n^2}-\frac{(-1)^n\pi}{in}-\frac{1}{n^2}\right)\\
              &= \frac{2(-1)^n-2}{2\pi n^2}=\frac{(-1)^n-1}{\pi n^2}
        \end{aligned}$$
        Also sind alle geraden Fourierkoeffizienten $0$ und alle ungeraden $\frac{2}{\pi n^2}$. Für $n=0$
        gilt:
        $$\hat{f}(0)= \frac{1}{2\pi}\int_{-\pi}^{\pi} |x|dx=\frac{1}{\pi}\int_0^\pi xdx=\frac{1}{2\pi}\pi^2
        =\frac{\pi}{2}$$
        Damit gilt:
        $$\begin{aligned}
              \frac{\pi^2}{3}&=\frac{1}{2\pi}\int_{-\pi}^{\pi} |f|^2dx=\sum_{n\in \Z}|\hat{f}(n)|^2\\
              &=\sum_{n\text{ ungerade}}\frac{4}{\pi^2|n^4|} + \frac{\pi^2}{4}\\
              &= \frac{8}{\pi^2}\sum_{n=0}^{\infty }\frac{1}{(2n+1)^4}+ \frac{\pi^2}{4}\\
              &\Rightarrow \sum_{n=0}^{\infty }\frac{1}{(2n+1)^4}=\left(\frac{\pi^2}{3}-\frac{\pi^2}{4}\right)
              \frac{\pi^2}{8}=\frac{\pi^4}{96}
        \end{aligned}$$
        Für den zweiten Punkt sehen wir:
        $$\begin{aligned}
              \sum_{n=1}^{\infty }\frac{1}{n^4}&= \sum_{n\text{ ungerade}}\frac{1}{n^4} +
              \sum_{n\text{ gerade}}\frac{1}{n^4}\\
              &= \frac{\pi^4}{96}+\frac{1}{2^4}\sum_{n=1}^{\infty}\frac{1}{n^4}\\
              \Rightarrow \frac{15}{16}\sum_{n=1}^{\infty}\frac{1}{n^4} &= \frac{\pi^4}{96}\\
              \sum_{n=1}^{\infty} \frac{1}{n^4}&=\frac{\pi^4}{90}
        \end{aligned}$$

        Wir verwenden wieder Parseval:
        $$\begin{aligned}
              \frac{1}{2\pi}\int_{-\pi}^{\pi} |f|^{2}dx &= \frac{1}{\pi}\int_0^{\pi} (x\pi-x^2)^{2}dx\\
              &= \frac{1}{\pi}\int_0^{\pi} x^2\pi^2-2\pix^3+x^4 dx\\
              &= \frac{\pi^2}{3\pi}\pi^3-\frac{2\pi}{4\pi}\pi^4+\frac{1}{5}\pi^5\\
              &= \pi^4\left(\frac{1}{3}-\frac{1}{2}+\frac{1}{5}\right)=\frac{\pi^4}{30}
        \end{aligned}$$
        Weiter wissen wir von Übungsblatt 9 Aufgabe 2, dass für die gegebene Funktion für $n\neq 0$
        $\hat{f}(n)=\frac{2i((-1)^n -1)}{\pi n^3}$, für $n$ gerade haben wir also $0$ als
        Fourierkoeffizient für $n$ ungerade haben wir:
        $$|\hat{f}(n)|^2=\left|\frac{2i((-1)^n -1)}{\pi n^3}\right|^2=\left(\frac{4}{\pi |n|^3}\right)^2
        =\frac{16}{\pi^2 |n|^6}$$
        Nach Parseval gilt damit:
        $$\frac{\pi^4}{30}=\frac{1}{2\pi}\int_{-\pi}^{\pi}|f|^2dx =\sum_{n \in \mathbb{Z}}|\hat{f}(n)|^2
        =\frac{16}{\pi^2}\sum_{n\text{ ungerade}}\frac{1}{|n|^6}=\frac{32}{\pi^2}
        \sum_{k\geq 0}\frac{1}{(2k+1)^6}$$
        Womit wegen $32\cdot 30=960$ die erste Behauptung folgt. Die zweite Behautptung folgt durch:
        $$\begin{aligned}
              \sum_{n=1}^{\infty}\frac{1}{n^6} &= \sum_{n\text{ ungerade}}\frac{1}{n^6} +
              \sum_{n\text{ gerade}}\frac{1}{n^6}\\
              &= \frac{\pi^6}{960}+\frac{1}{2^6}\sum_{n=1}^{\infty}\frac{1}{n^6}\\
              \Rightarrow \frac{63}{64}\sum_{n=1}^{\infty}\frac{1}{n^6} &= \frac{\pi^6}{960}\\
              \sum_{n=1}^{\infty} \frac{1}{n^6}&=\frac{\pi^6}{945}
        \end{aligned}$$
    \end{proof}

    \begin{problem}{3}
        Sei $\alpha \in \mathbb{R} \backslash \mathbb{Z}$. Zeigen Sie das Folgende.
        \begin{enumerate}[label = (\alph*)]
            \item Auf $[0,2\pi]$ gilt
            $$\frac{\pi}{\sin(\pi\alpha)}e^{i(\pi-x)\alpha}\sim \sum_{n\in\Z}\frac{e^{inx}}{n+\alpha}$$
            \item $$\sum_{n\in\Z}\frac{1}{(n+\alpha)^2}=\frac{\pi^2}{(\sin(\pi\alpha))^2}$$
        \end{enumerate}
    \end{problem}

    \begin{proof}
        Wir bestimmen die Fourierkoeffizienten für $f(x)=\frac{\pi}{\sin(\pi\alpha)}e^{i(\pi-x)\alpha}$
        $$\begin{aligned}
              \hat{f}(n) &= \frac{1}{2\pi}\int_{0}^{2\pi}\frac{\pi}{\sin(\pi\alpha)}e^{i(\pi-x)\alpha}
              e^{-inx}dx \\
              &= \frac{e^{i\pi\alpha}}{2\sin(\pi\alpha)}\int_{0}^{2\pi}e^{-ix(\alpha+n)}dx\\
              &= \frac{e^{i\pi\alpha}}{2\sin(\pi\alpha)}\frac{1}{-i(\alpha+n)}\left[e^{-ix(\alpha+n)}
              \right]_{0}^{2\pi}\\
              &= \frac{e^{i\pi\alpha}}{-2i\sin(\pi\alpha)(\alpha+n)}\left(e^{-2\pi i(\alpha+n)}-
              1\right)\\
              &=\frac{e^{i\pi\alpha}}{\sin(\pi\alpha)(\alpha+n)}
              \frac{1-e^{-2\pi i\alpha}}{2i}\\
              &= \frac{1}{\sin(\pi\alpha)(\alpha+n)}\cdot \frac{e^{i\pi\alpha}-e^{-i\pi\alpha}}{2i}
              = \frac{1}{\alpha+n}
        \end{aligned}$$
        Also haben wir wie gewünscht
        $$\frac{\pi}{\sin(\pi\alpha)}e^{i(\pi-x)\alpha}\sim \sum_{n\in\Z}\frac{e^{inx}}{n+\alpha}=
        \sum_{n\in\Z}\hat{f}(n)e^{inx}$$
        Damit gilt nach der Parseval-Isometrie 
        $$\begin{aligned}
              \frac{\pi^2}{(\sin(\pi\alpha))^2}
              &=\frac{1}{2\pi}\int_0^{2\pi}\frac{\pi^2}{(\sin(\pi\alpha))^2}\\
              &=\frac{1}{2\pi}\int_0^{2\pi}|f(x)|^2\\
              &= \sum_{n\in\Z}|\hat{f}(n)|^2 \\
              &= \sum_{n\in\Z}\frac{1}{(n+\alpha)^2}
        \end{aligned}$$
    \end{proof}

    \begin{problem}{4}
        Beweisen Sie das Riemann-Lebesgue-Lemma: Wenn $f\in L^1([-\pi,\pi])$, dann $\hat{f}(n)\to 0$ für
        $n\to\pm\infty$. \\
        (Hinweis: Betrachten Sie zuerst $f\in L^2([-\pi,\pi])$. Für $f\in L^1([-\pi,\pi])$ finden Sie
        $f_k\in L^2([-\pi,\pi])$ mit $\|f-f_k\|_1\to 0$ für $k\to\infty$)
    \end{problem}

    \begin{proof}
        Sei zuerst $f\in L^2$, dann gilt
        $$\lim_{N\to\infty}\sum_{n=-N}^{N}|\hat{f}(n)|^2=\sum_{n\in\Z}|\hat{f}(n)|^2 = \frac{1}{2\pi}
        \int_{-\pi}^{\pi}|f(x)|^2dx<\infty$$
        Damit gibt es für alle $\varepsilon>0$ ein $N$ sodass alle Fourierkoeffizienten mit einem Index, der
        betragsmäßig größer $N$ ist kleiner als $\varepsilon$, weil so die Konvergenz definiert ist. \\\\
        Sei nun $f\in L^1$ beliebig, dann definieren wir
        $$f_k(x):=
        \begin{cases}
            f(x) \qquad |f(x)|\leq k\\
            0 \qquad \text{sonst.}
        \end{cases}$$
        Diese sind alle integrierbar und sogar in $L^2$, denn sie sind beschränkt. \\\\
        Dann gilt
        $$\|f-f_k\|_1=\frac{1}{2\pi}\int_{-\pi}^\pi|f-f_k|dx\to 0$$
        Dabei gilt die letzte Konvergenz, denn $|f-f_k|$ ist eine integrierbare Funktion und wird von
        $|f|$ dominiert. Mit dem Satz der dominierten Konvergenz haben wir damit:
        $$\lim_{k\to\infty}\int_{-\pi}^\pi|f-f_k|dx=\int_{-\pi}^\pi\lim_{k\to\infty}|f-f_k|dx=0$$
        Damit gilt dann aber
        $$|\hat{f}(n)-\hat{f}_k(n)|= \left|\frac{1}{2\pi}\int_{-\pi}^{\pi} ({f}(x)-{f}_k(x))
        e^{-inx}dx\right|\leq \frac{1}{2\pi}\int_{-\pi}^{\pi} |{f}(x)-{f}_k(x)|dx\to 0$$
        Dieser Grenzwert ist unabhängig von $n$, also ist der Grenzwert gleichmäßig und damit
        gilt
        $$\lim_{n\to\infty}|\hat{f}(n)| = \lim_{n\to\infty}\lim_{k\to\infty}|\hat{f}_k(n)|=
        \lim_{k\to\infty}\lim_{n\to\infty}|\hat{f}_k(n)|=0$$
    \end{proof}
\end{document}