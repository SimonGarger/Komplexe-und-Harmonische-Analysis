\documentclass[11pt]{article}
\title{Zusammenfassung Numerik 2022 WS}
\author{Simon Garger}
\usepackage{amsmath}
\usepackage{amssymb}
\usepackage{amsfonts}
\usepackage{setspace}
\usepackage{fullpage}
\usepackage{blkarray}
\usepackage[a4paper,left=2.5cm,right=2.5cm,top=2cm,bottom=4cm,bindingoffset=5mm]{geometry}
\usepackage{graphicx}
\usepackage{wasysym}
\usepackage{mathtools}
\usepackage{titlesec}
\usepackage{xcolor}
\usepackage{bm}
\usepackage{enumitem}
\usepackage{amsbooka}
\usepackage{translit}
\usepackage{arabicore}
\usepackage{amsmath-2018-12-01}
\usepackage{amsmath-2018-12-01}


\newcommand{\N}{\mathbb{N}}
\newcommand{\Z}{\mathbb{Z}}
\newcommand{\R}{\mathbb{R}}
\newcommand{\C}{\mathbb{C}}
\renewcommand{\Re}{\mathfrak{Re}}
\renewcommand{\Im}{\mathfrak{Im}}


\newenvironment{problem}[2][Beispiel]{
    \begin{trivlist}
        \item[\hskip \labelsep {\bfseries #1}\hskip \labelsep {\bfseries #2.}] \itshape}{
    \end{trivlist}\normalshape
}

\begin{document}
    \begin{problem}{1}
        Seien $f_n: \Omega \rightarrow \mathbb{C}, n \in \mathbb{N}$, holomorphe Funktionen auf einer
        offenen Menge $\Omega \subset \mathbb{C}$, welche auf jeder kompakten Teilmenge von $\Omega$
        gleichmässig gegen eine Funktion $f: \Omega \rightarrow \mathbb{C}$ konvergieren. Aus der Vorlesung
        wissen wir, dass $f$ dann holomorph ist. Zeige ausserdem, dass sogar die Folgen der $k$-ten Ableitungen
        von $f$ gleichmässig gegen die $k$-te Ableitung von $f$ konvergieren, d.h. $f_n^{(k)}
        \rightarrow f^{(k)}(n \rightarrow \infty)$, auf jeder kompakten Teilmenge von $\Omega$.
    \end{problem}

    \begin{proof}
        Genauer argumentieren mit den Kompakten Mengen! \\\\
        Wir gehen wie beim Beweis von Satz 5.3 vor. Wieder nehmen wir oBdA an, dass die Folge der Funktionen
        auf ganz $\Omega$ konvergiert, andernfalls können wir es auf jeder kompakten Menge betrachten und
        zusammenstückeln. Wieder bilden wir $\Omega_\delta$, also die Menge, die $\Omega$ um einen
        $\delta$-Streifen verkleinert. Wieder zeigen wir:
        $$\sup_{z\in\Omega_\delta} |F^{(k)}|\leq \frac{k!}{\delta^k} \sup_{\zeta\in \Omega}|F(\zeta)|$$
        (Denn $F = f_n-f$ führt zum gewünschten Ergebnis, da die rechte Seite dann gegen $0$ konvergiert).
        Wir finden mit der Cauchy-Integralformel:
        $$\begin{aligned}
              |F^{(k)}(z)|
              &= \left|\frac{k!}{2\pi i}\int_{C_\delta (z)}\frac{F(\zeta)}{(\zeta-z)^{k+1}}\,d\zeta\right|\\
              &\leq \frac{k!}{2\pi}\int_{C_\delta (z)}\frac{|F(\zeta)|}{|(\zeta-z)^{k+1}|}\,d\zeta\\
              &\leq \frac{k!}{2\pi}\sup_{\zeta \in\Omega} |F(\zeta)|\frac{1}{\delta^{k+1}} 2\pi\delta\\
              &= \frac{k!}{\delta^{k}}\sup_{\zeta \in\Omega} |F(\zeta)|
        \end{aligned}$$
        Das gilt für alle $k$ und alle $\delta$. Zuletzt lassen wir noch $\delta$ gegen $0$ gehen und erhalten
        das Gewünschte.
    \end{proof}

    \begin{problem}{2}
        Benutze Kurvenintegrale in der komplexen Ebene, um das Integral
        $$
        \int_{-\infty}^{+\infty} \frac{1}{1+x^2} d x
        $$
        zu berechnen.
    \end{problem}

    \begin{proof}
        Zuerst bemerken wir, dass die Funktion $f(z) := \frac{1}{1+z^2}$ auf $\C\setminus\{\pm i\}$
        holomorph ist. \\\\
        Wir berechnen das Kurvenintegral über den oberen Halbkreis mit einem Schlüsselloch bei $i$.
        $$0 = \int_{-R}^{\varepsilon} \frac{1}{1+x^2}\, dx + \int_{\gamma_{\varepsilon}} f(z)\,dz +
        \int_{\varepsilon}^R \frac{1}{1+x^2}\, dx + \int_{\gamma_{R}} f(z)\,dz$$
        Wobei $\gamma_{\varepsilon}: t\mapsto \varepsilon e^{-it}+i$ mit $t\in [0,2\pi]$ und
        $\gamma_{R}: t\mapsto R e^{it}$ mit $t\in [0,\pi]$. Die Geraden zum Schlüsselloch wurden bereits
        weggelassen, da diese im Grenzwert verschwinden. Außerdem haben wir bei $\gamma_{\varepsilon}$
        das Intervall gleich wie im Grenzwert bis $2\pi$ zugelassen. Nun berechnen wir die Integrale:
        $$\left|\int_{\gamma_{R}} f(z)\,dz\right|\leq \int_{0}^\pi \frac{|Rie^{it}|}{|1+R^2 e^{2it}|}
        \,dt\leq \int_{0}^\pi \frac{R}{R^2-1}\,dt= \frac{R\pi}{R^2-1} \to 0$$
        für $R\to\infty$. \\\\
        Weiter finden wir:
        $$\int_{\gamma_{\varepsilon}} f(z)\,dz = \int_{0}^{2\pi}\frac{1}{1+\varepsilon^2 e^{-2it} +
        2\varepsilon i e^{-it} -1}\varepsilon(-i)e^{-it}\,dt =-i\int_{0}^{2\pi}\frac{1}{\varepsilon e^{-it} +
        2i}\,dt \to -i\frac{1}{2i}2\pi = -\pi $$
        Für $\varepsilon\to 0$.\\\\
        Insgesamt haben wir damit also im Grenzwert von $\varepsilon\to 0$ und $R\to\infty$:
        $$\int_{-\infty}^{+\infty} \frac{1}{1+x^2} d x = - \int_{\gamma_{\varepsilon}} f(z)\,dz = \pi$$
        Anmekrung aus den Übung: $\varepsilon$ muss garnicht gegen $0$ gehen. Geht leichter
        mit Partialbruchzerlegung. Und es geht auch recht cool mit der Cauchy-Integralformel.
    \end{proof}

    \begin{problem}{3}
        Beweise: Das Bild einer nicht-konstanten ganzen Funktion ist dicht in $\mathbb{C}$.
    \end{problem}

    \begin{proof}
        Nehmen wir an $f(\C)$ ist nicht dicht. Dann finden wir ein $a$, sodass $B_\varepsilon (a)\cap f(\C) =
        \emptyset$. Nun betrachten wir die Funktion $\frac{1}{f(z)-a}$, welche natürlich ganz und
        betragsmäßig kleiner gleich $\frac{1}{\varepsilon}$ ist, also ist sie konstant und somit ist $f(z)$
        konstant, was ein Widerspruch zur Annahme ist.
    \end{proof}

    \begin{problem}{4}
        Betrachte die durch die folgenden Potenzreihen definierten Funktionen:
        $$
        f(z):=\sum_{n=0}^{\infty} z^n, \quad g(z):=\sum_{n=0}^{\infty} z^{2^n} .
        $$
        \begin{enumerate}[label = (\alph*)]
            \item Zeige: $f$ und $g$ definieren auf $B_1(0)$ holomorphe Funktionen.
            \item Kann $f$ holomorph auf eine offene Menge $\Omega$ mit $B_1(0) \subsetneq \Omega \subset \mathbb{C}$
        fortgesetzt werden?
            \item Kann $g$ holomorph auf eine offene Menge $\Omega$ mit $B_1(0) \subsetneq \Omega \subset \mathbb{C}$
        fortgesetzt werden?
        \end{enumerate}
    \end{problem}

    \begin{proof}
        \begin{enumerate}[label = (\alph*)]
            \item Wir wissen aus Satz 2.6, dass eine Potenzreihe auf ihrem Konvergenzradius eine holomorphe
            Funktion definiert. Für den Konvergenzradius der ersten Reihe gilt:
            $$R = \frac{1}{\limsup_{n\to\infty}\sqrt[n]{1}} = 1$$
            Wir schreiben die zweite Reihe zu $\sum_{n\geq 0} a_n z^{n}$ mit $a_n = 1$, wenn ein $k$
            existiert sodass $2^k = n$ gilt und $0$ sonst. Damit finden wir den Konvergenzradius:
            $$R = \frac{1}{\sqrt[n]{\limsup_{n\to\infty}|a_n|}} = 1$$
            Also sind beide Funktionen auf $B_1(0)$ holomorph.
            \item Auf der Einheitskreisscheibe gilt $f(z) = \frac{1}{1-z}$. Diese Funktion ist
            aber überall außer in $1$ holomorph und somit können wir die Funktion auf $\Omega = \C\setminus
            \{0\}$ fortsetzen.
            \item Hingegen kann man diese Funktion nicht holomorph fortsetzen. Wir betrachten die
            Punkte $z = re^{i\frac{2\pi p}{2^k}}$ mit $p,k\in\N$. Nun gilt:
            $$g(z)=\sum_{n=0}^{\infty} r^{2^n}e^{i2\pi p 2^{n-k}} = \sum_{n=0}^{k} r^{2^n}e^{i2\pi p 2^{n-k}}
            +\sum_{n=k+1}^{\infty} r^{2^n}e^{i2\pi p 2^{n-k}}\leq \sum_{n=k+1}^{\infty} r^{2^n} \to \infty$$
            für $r\to 1$. Nun liegt diese Menge allerdings dicht auf dem Einheitskreis. Denn sei
            $\varphi$ ein beliebiger Winkel und $\varepsilon>0$ beliebig. Dann finden wir:
            $$\frac{2\pi}{2^{\min(n| \frac{2\pi}{2^n}<\varepsilon)}}\cdot p$$
            wobei wir $p$ so wählen, dass unsere Approximation bestmöglich wählen. Somit kann man die Potenzreihe
            aber nicht einmal auf den Abschluss fortsetzen.
        \end{enumerate}
    \end{proof}

    \begin{problem}{5}
        Sei $f: \overline{B_1(0)} \rightarrow \mathbb{C}\setminus\{0\}$ stetig und holomorph auf $B_1(0)$. Zeige:
        Falls $|f(z)|=1$ für alle $z \in \partial B_1(0)$, dann ist $f$ konstant.
        (Hinweis: Betrachte die Fortsetzung von $f$ auf ganz $\mathbb{C}$, welche durch Reflektion am
        Einheitskreis definiert ist.)
    \end{problem}

    \begin{proof}
        Wir betrachten die Funktion
        $$F(z):=
        \begin{cases}
            f(z)\qquad &z\in B_1(0)\\
             \frac{1}{\overline{f(1/\overline{z})}}\qquad &z\notin B_1(0)
        \end{cases}$$
        Da $f$ nirgendwo verschwindet, ist $f$ außerhalb des Einheitskreises holomorph.
        Innerhalb des Einheitskreises ist die Funktion natürlich immernoch holomorph. \\
        Weiter gilt für Punkte $z_0$ auf dem Einheitskreis:
        $$F(z_0) = \frac{1}{\overline{f(1/\overline{z_0})}} =\frac{1}{\overline{f(z_0)}}=f(z_0)$$
        Also setzt sich diese Funktion stetig am Rand fort und ist somit überall holomorph.\\\\
        Nun nimmt die Funktion auf der Einheitskreisscheibe natürlich das Maximum und das Minimum an
        und somit finden wir:
        $$|F(z)|\leq \max_{z\in \overline{B_1(0)}} \max(|f(z)|, |1/f(z)|) $$
        Die Funktion ist somit ganz und beschränkt, also ist sie konstant.\\\\
        Holomorphie von Dreiecken über den Rand muss man eigentlich noch zeigen.
    \end{proof}
\end{document}