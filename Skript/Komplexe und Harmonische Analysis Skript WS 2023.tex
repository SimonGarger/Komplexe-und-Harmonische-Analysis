\documentclass[a4paper, 12pt]{article}          % Paper and font size % Language setting
% Set margins
\usepackage[top=2.5cm,bottom=2.5cm,left=2cm,right=2cm,marginparwidth=1.75cm]{geometry}

% Load useful packages
\usepackage{graphicx}                           % Graphics
\usepackage{amsmath}                            % Math
\usepackage{xcolor}                             % Link color
\definecolor{custom-blue}{RGB}{0,99,166}
\usepackage[T1]{fontenc}                        % Special characters
\usepackage{fancyhdr}                           % Load header, footer package
\usepackage[ngerman]{babel}                     % Sprache


\PassOptionsToPackage{dvipsnames}{xcolor}       % Packages Anton
\usepackage{tikz}
\usepackage{amsfonts}
\usepackage{enumitem}
\usepackage{amssymb}
\usepackage{marvosym}
\usepackage{mathtools}
\usepackage{empheq}
\usepackage{cancel}
\usepackage{harpoon}
\usetikzlibrary{graphs,tikzmark,calc,arrows,arrows.meta,angles,math,decorations.markings}
\usepackage{pgfplots}
\usepackage{booktabs}
\usepackage{framed}
\usepackage[hyperref,amsmath,thmmarks,framed]{ntheorem}
\usepackage[colorlinks=true, linkcolor=magenta, psdextra, pdfencoding=auto]{hyperref}
\usepackage[capitalize,nameinlink]{cleveref}
\usepackage{tcolorbox}
\usepackage{lmodern}
\usepackage{faktor}
\usepackage{mathpunctspace}


\usepackage{lastpage}                           % Footer note
\usepackage{lipsum}
\usepackage{amsfonts}
\usepackage{amsbooka}

\pagestyle{myheadings}                          % Own header
\pagestyle{fancy}                               % Own style
\fancyhf{}                                      % Clear header, footer

\setlength{\headheight}{30pt}                   % Set header hight
\renewcommand{\headrulewidth}{0.5pt}            % Top line
\renewcommand{\footrulewidth}{0.5pt}            % Bottom line
\setlength{\headsep}{1cm}

\fancyhead[L]{\includegraphics[width=3cm]{Uni_Logo_2016.jpg}} % Header left
\fancyhead[C]{}                                 % Header center
\fancyhead[R]{Komplexe und Harmonische Analysis}       % Header right
\fancyfoot[L]{}                                 % Footer left
\fancyfoot[C]{}                                 % Footer center
\fancyfoot[R]{\thepage/\pageref{LastPage}}      % Footer right

% Environments
% \newtheorem*{proof}{\textit{Beweis:}}
\DeclareRobustCommand{\qed}{%
    \ifmmode % if math mode, assume display: omit penalty etc.
    \else \leavevmode\unskip\penalty9999 \hbox{}\nobreak\hfill
    \fi
    \quad\hbox{\qedsymbol}}
\newcommand{\openbox}{\leavevmode
\hbox to.77778em{%
    \hfil\vrule
    \vbox to.675em{\hrule width.6em\vfil\hrule}%
    \vrule\hfil}}
\newcommand{\qedsymbol}{\openbox}
\newenvironment{proof}[1][\proofname]{\par
\normalfont\trivlist
\item[\hskip\labelsep\itshape
    #1.]\ignorespaces
}{%
\qed\endtrivlist
}

\newenvironment{bemerkung}[1][\textit{Bemerkung}]{\par
\normalfont\trivlist
\item[\hskip\labelsep\itshape
    #1.]\ignorespaces
}{%
\qed\endtrivlist
}

\newenvironment{beispiel}[1][\textit{Beispiel}]{\par
\normalfont\trivlist
\item[\hskip\labelsep\itshape
    #1.]\ignorespaces
}{%
    \qed\endtrivlist
}

\newcommand{\proofname}{Proof}
\makeatother

\definecolor{blau}{HTML}{29b0c2}
\definecolor{rosa}{HTML}{c48494}
\definecolor{weiss}{HTML}{FFFFFF}

\theoremstyle{break}
\theoremseparator{:\smallskip}
\theoremindent=1em
\theoremheaderfont{\kern-1em\normalfont\bfseries}
\theorembodyfont{\normalfont}
\theoreminframepreskip{0em}
\theoreminframepostskip{0em}
\theoremsymbol{}
\newtcbox{\theoremBox}{colback=rosa!17,colframe=rosa!87,boxsep=0pt,left=7pt,right=7pt,top=7pt,bottom=7pt}
\def\theoremframecommand{\theoremBox}

\newshadedtheorem{theo}{Theorem}[section]

\newshadedtheorem{satz}[theo]{Satz}
\newcommand{\satzautorefname}{Satz}
\theoremstyle{nonumberbreak}
\newshadedtheorem{nonumbersatz}{Satz}
\theoremstyle{break}
\newshadedtheorem{lemma}[theo]{Lemma}
\newcommand{\lemmaautorefname}{Lemma}
\newshadedtheorem{korollar}[theo]{Korollar}
\newcommand{\korollarautorefname}{Korollar}
\newshadedtheorem{folgerung}[theo]{Folgerung}
\newcommand{\folgerungautorefname}{Folgerung}
\newshadedtheorem{proposition}[theo]{Proposition}
\newcommand{\propositionautorefname}{Proposition}
\newtcbox{\definBox}{colback=blau!17,colframe=blau!94,boxsep=0pt,left=7pt,right=7pt,top=7pt,bottom=7pt}
\def\theoremframecommand{\definBox}
\newshadedtheorem{definition}[theo]{Definition}
\newcommand{\definautorefname}{Definition}
\newtcbox{\miscBox}{colback=gray!17,colframe=gray!80}

\def\theoremframecommand{\miscBox}
\newshadedtheorem{bemerkung}[theo]{Bemerkung}
\newcommand{\bemerkungautorefname}{Bemerkung}
\newshadedtheorem{beispiel}[theo]{Beispiel}

\newtcbox{\warnBox}{colback=red!17,colframe=red!80}
\def\theoremframecommand{\warnBox}
\theoremstyle{nonumberbreak}

\hypersetup{colorlinks=true, allcolors=custom-black}



\newcommand{\chapter}[1]{\pagebreak \huge\textbf{\vskip 2cm #1} \normalsize}
\newcommand{\N}{\mathbb{N}}
\newcommand{\Z}{\mathbb{Z}}
\newcommand{\R}{\mathbb{R}}
\newcommand{\C}{\mathbb{Ce}}
\renewcommand{\Re}{\mathfrak{Re}}
\renewcommand{\Im}{\mathfrak{Im}}


%%% Begin document %%%
\begin{document}
\begin{figure}                                  % Include Logo
    \flushleft
\includegraphics[width=0.6\textwidth]{Uni_Logo_2016.jpg}
\end{figure}

% Set title, author, date
\title{Komplexe und Harmonische Analysis WS2023}
\author{Simon Garger - simon.garger@gmail.com}
\date{\today{}, Wien}

\maketitle
\thispagestyle{empty}
\pagebreak% Include contents
\tableofcontents
\pagebreak


\chapter{Komplexe Analysis}
\\\\
    \begin{bemerkung}
        Im Folgenden wird immer $z_k=x_k+iy_k=r_k\cdot e^{i\theta_k}$ gelten. Außerdem wird im Allgemeinen
        $\Omega$ eine Teilmenge von $\C$ sein.
    \end{bemerkung}

\section{Grundlagen}

    \begin{definition}[Komplexe Zahlen]
        Wir definieren den komplexen Zahlenkörper als die Menge
        $$\C:= \{z=x+iy|x,y\in \R\}$$
        Dabei bezeichnet $i$ die komplexe Einheit mit $i^2=-1$.
        Man nennt $x=\Re(z)$ den Realteil und $y=\Im(z)$ den Imaginärteil.
        Der Betrag einer komplexen Zahl wird geschrieben durch $|z| = \sqrt{x^2+y^2}$.
        Wir definieren $\bar{z}:= x-iy$ als das komplex Konjugierte von $z$.
    \end{definition}

    \begin{proposition}
        Die komplexen Zahlen erfüllen folgende Rechenregeln:
        \begin{enumerate}
            \item $z_1+z_2 = (x_1+x_2)+i(y_1+y_2)$ \\(Interpretation: Vektoraddition)
            \item $z_{1}z_2 = (x_1x_2-y_1y_2)+i(x_1y_2+x_2y_1)=r_{1}r_2\cdot e^{i(\theta_1 +\theta_2)}$
            \\(Interpretation: Streckung und Drehung)
            \item $|z_1+z_2|\leq |z_1|+|z_2|$
            \item $|z|^2=z\bar{z}\Rightarrow \dfrac{1}{z} = \dfrac{\bar{z}}{|z|^2}$
        \end{enumerate}
    \end{proposition}

    \begin{definition}
        Da $\C\cong\R^2$ gilt, können wir die topologischen Eigenschaften von $\R^2$ in $\C$ übertragen
        und werden auch im Weiteren öfter die komplexen Zahlen mit der zweidimensionalen Zahlenebene
        identifizieren. Weiter werden wir eine Teilmenge von $\C$ meist mit $\Omega$ bezeichnen.
        Wir erhalten damit:
        \begin{enumerate}
            \item Die komplexen Zahlen sind vollständig, es konvergiert also jede Cauchy-Folge.
            \item Eine Folge konvergiert in $\C$ genau dann, wenn Real- und Imaginärteil konvergieren und genau
            dann, wenn es eine Cauchy-Folge ist.
            \item Das Innere von $\Omega$ ist $\Omega^\circ := \{a|\exists B_r: a\in B_r\subset \Omega\}$
            \item Wir nennen $\Omega$ kompakt, sofern die Menge beschränkt und abgeschlossen ist.
            \item Wir nennen $\Omega$ offen (abgeschlossen) zusammenhängend, wenn für $\Omega_1,\Omega_2$ offen
            (abgeschlossen) gilt:
            $$\Omega =\Omega_1\cup\Omega_2\quad\land\quad \emptyset = \Omega_1\cap \Omega_2\quad\Rightarrow
            \quad \Omega_1 = \emptyset\quad\lor \quad \Omega_2=\emptyset$$
            \item Wir nennen eine Funktion $f$ stetig, wenn gilt:
            $$\lim_{n\to \infty} z_n = z\quad \Rightarrow \quad \lim_{n\to \infty} f(z_n) = f(z)$$
        \end{enumerate}
    \end{definition}


\section{Holomorphie}

    \begin{definition}
        Sei $z_0\in \Omega$, dann nennen wir $f$ holomorph (komplex differenzierbar) in $z_0$ falls
        $$\lim_{h\to 0}\dfrac{f(z_0+h)-f(z_0)}{h}=f'(z_0)\in \C$$
        existiert. \\\\
        Eine Funktion heißt auf $\Omega$ holomorph, wenn sie in jedem Punkt $z\in \Omega$ holomorph ist oder
        einfach "ganz", falls sie für alle $z\in \C$ holomorph ist.
    \end{definition}

    \begin{beispiel}
        Beispiele für holomorphe Funktionen sind:
        \begin{itemize}
            \item Konstante Funktionen
            \item Potenzfunktionen
            \item Polynomfunktionen
            \item Potenzreihen
        \end{itemize}
        Hingegen ist $f(z) = \bar{z}$ nicht holomorph, da
        $$\lim_{h\to 0} \dfrac{f(z+h)-f(z)}{h} = \lim_{h\to 0} \dfrac{\bar{h}}{h}$$
        und der letzte Term existiert nicht (für $h$ reell $1$ für $h$ rein imaginär $-1$).
    \end{beispiel}

    \begin{proposition}[Cauchy-Riemann Gleichungen]
        Sei $f$ eine auf $\Omega$ holomorphe Funktion. Identifizieren wir nun $f(z) = u(z)+iv(z)$, wobei
        $u,v:\C\cong\R^2\to \R$ sind, so gelten folgende Beziehungen:
        $$\begin{aligned}
              \dfrac{\partial u}{\partial x}(x,y) &= \dfrac{\partial v}{\partial y}(x,y)\\
              \dfrac{\partial u}{\partial y}(x,y) &= -\dfrac{\partial v}{\partial x}(x,y)
        \end{aligned}$$
    \end{proposition}

    \begin{proof}
        Wir definieren für ein beliebiges $x+iy=z_0\in \Omega$:
        $$\lim_{h\to 0}\dfrac{f(z_0+h)-f(z_0)}{h}=f'(z_0)=: a+ib$$
        Nun differenzieren wir nur partiell (der obige Limes beschreibt ja eine beliebige Nullfolge, also
        können wir auch die entlang der Achsen betrachten), indem wir die obige $u,v$ Identifikation und
        $h=(h_1,h_2)$ verwenden:
        $$\begin{aligned}
              a+ib &= \lim_{h_1 \to 0}\dfrac{f(z_0+h_1)-f(z)}{h_1} \\
              &=\lim_{h_1 \to 0}\dfrac{u(z_0+h_1)+iv(z_0+h_1)-u(z_0)-iv(z_0)}{h_1}\\
              &=\lim_{h_1 \to 0}\dfrac{u(z_0+h_1)-u(z_0)}{h_1} +\lim_{h_1 \to 0}\dfrac{i(v(z_0+h_1)-v(z_0))}{h_1}\\
              &=\dfrac{\partial u}{\partial x}(a,b) +i\dfrac{\partial v}{\partial x}(a,b)
        \end{aligned}$$
        Genauso finden wir:
        $$\begin{aligned}
              a+ib &= \lim_{h_2 \to 0}\dfrac{f(z_0+ih_2)-f(z_0)}{ih_2} \\
              &=\lim_{h_2 \to 0}\dfrac{u(z_0+ih_2)+iv(z_0+ih_2)-u(z_0)-iv(z_0)}{ih_2}\\
              &=\lim_{h_2 \to 0}\dfrac{u(z_0+ih_2)-u(z_0)}{ih_2} +\lim_{h_2 \to 0}\dfrac{i(v(z_0+ih_2)-v(z_0))}{ih_2}\\
              &=\frac{1}{i}\dfrac{\partial u}{\partial y}(a,b) +\dfrac{\partial v}{\partial y}(a,b)
        \end{aligned}$$
        Damit folgt (unter Berücksichtigung von $\frac{1}{i}=-i$):
        $$\begin{aligned}
              &a = \dfrac{\partial u}{\partial x}(x,y) = \dfrac{\partial v}{\partial y}(x,y)\\
              &b = \dfrac{\partial v}{\partial x}(x,y) = -\frac{\partial u}{\partial y}(x,y)
        \end{aligned}$$
    \end{proof}

    \begin{proposition}
        $$\dfrac{\partial }{\partial z} := \dfrac{1}{2}\left(\dfrac{\partial}{\partial x}
        +\dfrac{1}{\dfrac{\partial}{\partial y}\right)$$
        $$\dfrac{\partial }{\partial z} := \dfrac{1}{2}\left(\dfrac{\partial}{\partial x}
        -\dfrac{1}{\dfrac{\partial}{\partial y}\right)$$
        Dann gilt: $f$ holomorph, falls:
        $$\dfrac{\partial f}{\partial z}=0$$
        $$\dfrac{\partial f}{\partial t}=f'(z) = 2\dfrac{\partial u}{\partial z}$$
    \end{proposition}

    \begin{bemerkung}
        $F$ ist reell differenzierbar in $(x,y)\in \R^2 \Leftrightarrow \exists J_f(x_0,y_0)
        :\R^2\to\R^2$ linear, sodass für $H=(h_1,h_2)$
        $$\lim_{\|H\|\to 0}\frac{F(z_0+H) - F(z_0)-J_f(z_0)H}{\|H\|} = 0$$
        gilt. Dabei geht es nur um die Länge von $H$ und nicht wie bei der Holomorphie
        um die Existenz des Grenzwertes in allen Richtungen.
    \end{bemerkung}


    \begin{beispiel}
        Die Funktion, die komplex konjugiert, entspricht im reellen:
        $$F(x,y) =  \begin{pmatrix}
                        x\\-y
                    \end{pmatrix}$$
        Diese Abbildung ist schon linear, also ist sie insbesondere reell
        differenzierbar.
    \end{beispiel}

    \begin{proposition}
        $f$ holomorph $\Rightarrow F$ reell differenzierbar mit
        $|\det J_f(x_0,y_0)| = |f'(z_0)|^2$
    \end{proposition}

    \begin{proof}
        $f$ holomorph impliziert
        $$f(z+h) = f(z)+f'(z)h+hR(h)$$
        mit $R(h)\to 0$ für $h\to 0$.\\\\
        $$J_f\cdot H =
        \begin{pmatrix}
            \dfrac{du}{dx} & &\dfrac{du}{dy}\\
            \\ \dfrac{dv}{dx} & &\dfrac{dv}{dy}
        \end{pmatrix}\begin{pmatrix} h_1\\h_2
        \end{pmatrix} =
        \begin{pmatrix}
            \dfrac{du}{dx}h_1 + \dfrac{du}{dy}h_2 \\ \\
            \dfrac{dv}{dx}h_1 + \dfrac{dv}{dy}h_2
        \end{pmatrix}$$
        Nun identifizieren wir $\R^2\cong \C$:
        $$= \left(\dfrac{du}{dx}h_1+\dfrac{du}{dy}h_2\right)+i \left(\dfrac{dv}{dx}h_1+\dfrac{dv}{dy}h_2\right)$$
        $$= \left(\dfrac{\partial u}{\partial x}- i\dfrac{\partial u}{\partial y}\right) (h_1+ih_2)$$
        Wobei die letzte Gleichheit aus den Cauchy-Riemann Gleichungen folgt.\\\\
        Gesamt erhalten wir damit wie gewünscht:
        $$|\det J_f| = \left|\dfrac{du}{dx}\dfrac{dv}{dy}-\dfrac{du}{dy}\dfrac{dv}{dx}\right|
        = \left|\left(\dfrac{du}{dx}\right)^2-\left(\dfrac{du}{dy}\right)^2\right| =
        \left|\left(\dfrac{\partial u}{\partial x}- i\dfrac{\partial u}{\partial y}\right)\right|^2
        = |f'|^2 $$
    \end{proof}

\begin{satz}
    Sei $F$ reell differenzierbar mit
    $$\dfrac{\partial u}{\partial x} =
    \dfrac{\partial v}{\partial y}, \dfrac{\partial u}{\partial y} = - \dfrac{\partial v}{\partial x}$$
    mit $u,v\in C^1 (\Omega)$, dann ist $f$ auch holomorph.
\end{satz}

\begin{proof}
    Wir wollen zeigen: $\exists a \in \C$ sodass
    $$f(z+h)-f(z)-ah = hR(h)$$
    mit $R(h)\to 0$ für $h\to 0$. \\\\
    Wir finden:
    $$u(z+h)-u(z) = \dfrac{du}{dx}h_1 + \dfrac{du}{dy}h_2 + h_1R^1_u(h_1) + h_2R^2_u(h_2)$$
    und auch
    $$v(z+h)-v(z) = \dfrac{dv}{dx}h_1 + \dfrac{dv}{dy}h_2 + h_1R^1_v(h_1) + h_2R^2_v(h_2)$$
    Mit Cauchy-Riemann folgt die Behauptung. Wir definieren mit diesen:
    $$a:= \dfrac{du}{dx} + i\left(-\dfrac{du}{dy}\right)$$
    Denn damit erhalten wir:
    $$\begin{aligned}
          &f(z+h)-f(z)-ah = u(z+h)-u(z)+i(v(z+h)-v(z)) - \left(\dfrac{du}{dx} + i\left(-\dfrac{du}{dy}\right)\right)h\\
          &= \dfrac{du}{dx}h_1 + \dfrac{du}{dy}h_2 + R_u h +
          i\left(\dfrac{dv}{dx}h_1 + \dfrac{dv}{dy}h_2 + R_v h\right)-\dfrac{du}{dx}h_1-\dfrac{du}{dy}h_2
          -i\left(-\dfrac{du}{dy}h_1 + \dfrac{du}{dx}h_2\right)\\
          &= h_1R^1_u(h_1) + h_2R^2_u(h_2) + h_1R^1_v(h_1) + h_2R^2_v(h_2) =: hR(h)
    \end{aligned}$$
    Dabei geht $R(h)$ natürlich wie gewünscht gegen $0$ für $h\to 0$, weil die $R_i^j(h_k)$ alle gegen
    $0$ gehen.
\end{proof}

\section{Potenzreihen}
\begin{definition}
    Eine Potenzreihe ist ein Ausdruck der Form
    $$\sum_{n=0}^\infty a_n z^n$$
    wobei $a_n \in \C$.\\\\
    Absolute Konvergenz bedeutet
    $$\sum_{n=0}^\infty |a_n| |z|^n<\infty$$
\end{definition}

\begin{beispiel}
    $\quad$
    \begin{enumerate}[label = (\roman{enumi})]
        \item geoemtrische Reihe $\sum_{n=0}^\infty z^n$\\
        konvergiert auf $B_1(0) = \{|z|<1\}$ und dort gilt:
        $$\sum_{n=0}^\infty z^n=\dfrac{1}{1-z}$$
        Denn es gilt:
        $$(1-z) \sum_{n=0}^N z^n = (1-z^{N+1})\to 1 \qquad (\text{für  }N\to\infty)$$
        \item Die Exponentialreihe:
        $$e^z = \sum_{n=0}^\infty \frac{z^n}{n!}$$
    \end{enumerate}
\end{beispiel}

\begin{satz}
    Sei $\sum_{n=0}^\infty a_n z^n$ gegeben, dann existiert ein $0\leq R\leq +\infty$
    "Konvergenzradius", sodass
    \begin{enumerate}[label = (\roman{enumi})]
        \item $|z|<R$, dann konvergiert die Folge absolut
        \item $|z|>R$, dann divergiert die Folge
    \end{enumerate}
    Es gilt
    $$R = \dfrac{1}{L} \quad\text{mit}\quad L = \limsup_{n\to \infty} |a_n|^{1/n}$$
    Für $L=0$ definieren wir $R=\infty$ und für $L=\infty$ $R=0$.
\end{satz}

\begin{bemerkung}
    Das Verhalten auf dem Rand $B_R(0) = \{|z| =R\}$ kann kompliziert sein. Hier kann
    es eine Mischung aus Konvergenz und Divergenz geben.\\\\
    $e^z$ konvergiert auf ganz $\C$.\\\\
    Äquivalent ist das Quotientenkriterium (Übung $5$).\\\\
    Bei diesen Beweisen ändert sich verhältnismäßig wenig im Umstieg von reell auf
    komplex, da wir über absolute Konvergenz, also Konvergenz reeller Zahlen reden.
\end{bemerkung}

\begin{proof}
    $\quad$
    \begin{enumerate}
        \item $L=0\Rightarrow R=\infty$:
        $$\forall z\in \C, \exists N>0, \exists 0\leq r<1,\forall n>N: |a_n||z|^n = (a_n^{1/n}|z|)^n<r^n$$
        Daraus folgt:
        $$\sum_{n=N}^M |a_n||z|^n <\sum_{n=N}^M r^n <\sum_{n^N} ^\infty <\infty$$
        \item $L=\infty \Rightarrow R=0$ :
        $$\forall z\in \C,\exists  N, (a_{n_k})_{k\in\N}, r>1, \forall n_k>N: |a_{n_k}||z|^n \geq r >1 $$
        damit divergiert die Reihe.
        \item $0<L<\infty$: Sei $R=\frac{1}{L}$ und $|z|<R$. Dann gilt
        $$\exists \varepsilon >0, 0\leq r<1: (L+\varepsilon )|z|=r<1 $$
        Damit folgt:
        $$\exists N, \forall n\geq N: |a_n|^{1/n}\leq L+\varepsilon$$
        Damit befinden wir uns wieder im Fall des ersten Unterpunkts und können somit wieder
        die Konvergenz innerhalb von $R$ beschließen.
        \item Ist andererseits $|z|>R$, dann folgt:
        $$\exists \varepsilon >0, r>1: (L-\varepsilon )|z|=r>1 $$
        Es existiert eine Teilfolge $(a_{n_k})_{k\in\N}$, sodass:
        $$\exists N, \forall k\geq N: |a_{n_k}|^{1/n_k}> L-\varepsilon$$
        Damit folgt:
        $$\sum_{n=0}^\infty |a_n| |z|^n\geq \sum_{k=0}^\infty |a_{n_k}||z|^{n_k} >
        \sum_{k=0}^\infty r^{n_k}$$
        wobei nun der letzte Term divergiert.
    \end{enumerate}
\end{proof}

\begin{satz}
    Eine Potenzreihe $\sum_{n=0}^\infty a_n z^n$ definiert eine innerhalb ihres Konvergenzradius
    $R$ holomorphe Funktion $f(z) = \sum_{n=0}^\infty a_n z^n$.
    Außerdem gilt:
    $$f'(z) = \sum_{n=0}^\infty n a_n z^{n-1}$$
    und $f'(z)$ konvergiert absolut auf $B_R(0)$.
\end{satz}

\begin{bemerkung}
    $f$ und $f'$ haben den gleichen Konvergenzradius, da $\sqrt[n]{n} = 1$. Insbesondere
    folgt, dass Potenzreihen innerhalb ihres Konvergenzradius unendlich oft komplex
    differenzierbar sind.
\end{bemerkung}

\begin{beispiel}$\quad$
    \begin{enumerate}
        \item $e^z$ ist ganz mit
        $$\dfrac{\partial }{\partial z}(e^z) = \sum_{n=0}^\infty \dfrac{n z^{n-1}}{n!} =
        \sum_{n=0}^\infty \dfrac{z^n}{n!} = e^z$$
        \item Die trigonometrischen Funktionen $\sin$ und $\cos$ sind ganz:
        $$\cos(z) = \sum_{n=0}^\infty (-1)^n\dfrac{z^{2n}}{(2n)!}= \dfrac{e^{iz}+e^{-iz}}{2}$$
        $$\sin(z) = \sum_{n=0}^\infty (-1)^n\dfrac{z^{2n+1}}{(2n+1)!} = \dfrac{e^{iz}-e^{-iz}}{2i}$$
        $$\Rightarrow \qquad e^{iz} = \cos(z) + i \sin(z)$$
    \end{enumerate}
\end{beispiel}

\begin{proof}
    Sei $f(z) = S_N(z) + E_N(z)$ mit $S_N(z) = \sum_{n=0}^N a_n z^n$ und $E_N(z) = f(z)- S_N(z)$
    und $g(z) = \sum_{n=0}^\infty  na_n z^n$.\\\\
    In der vorigen Bemerkung wurde bereits erwähnt, dass der Konvergenzradius von
    $g$ gleich dem von $f$, also $R$ ist.\\\\
    Wir folgern:
    $$\begin{aligned}
          \dfrac{f(z+h)-f(z)}{h}-g(z) &= \dfrac{S_N(z+h)-S_N(z)}{h}-g(z) +\dfrac{E_N(z+h)-E_N(z)}{h}\\
          &= \dfrac{S_N(z+h)-S_N(z)}{h}-S'_N(z)+S'_N(z)-g(z) +\dfrac{E_N(z+h)-E_N(z)}{h}
    \end{aligned}$$
    Per Definition gilt:
    $$\lim_{h\to 0}\dfrac{S_N(z+h)-S_N(z)}{h}-S'_N(z) = 0$$
    Weiter gilt, weil die Potenzreihe auf dem Konvergenzradius konvergiert:
    $$\lim_{N\to\infty }S'_N-g(z) = \lim_{N\to \infty}\sum_{n=N+1}^\infty n a_n z^{n-1} = 0$$
    Zuletzt sehen wir noch:
    $$\lim_{h\to 0}\dfrac{|E_N(z+h)-E_N(z)|}{|h|} = \lim_{h\to 0}\dfrac{|\sum_{n=N+1}^\infty a_n(z+h)^n- a_nz^n|}{|h|}
    \leq \sum_{n=N+1}^\infty n |a_n| |z|^{n-1}\to 0$$
für $N\to \infty$.
\end{proof}

\section{Kurvenintegrale}

    \begin{definition}
        Eine Abbildung
        $$z:[a,b]\to\C$$
        ist eine \underline{parametrisierte Kurve}.\\\\
        Wir sagen $z$ ist glatt oder $C^1$, falls für alle $t\in[a,b]$ die
        Ableitung $z'(t)$ existiert, stetig ist und $z'(t)\neq 0$ gilt. \\\\
        Stückweise glatt heißt eine Kurve, wenn $\exists a=a_0<a_1 <\dots <a_n=b$
        mit $z|_{[a_j,a_{j+1}]}$ glatt. \\\\
        Eine Parametrisierung $\tilde{z}:[c,d]\to \C$ heißt äquivalent zu $z$ falls
        $\exists t:[c,d]\to[a,b]$ eine Bijektion mit $t'(s)\neq 0$ so dass
        $$\tilde{z}(s)=z(t(s))$$
        Die Familie aller äquivalenten Parametrisierungen definiert eine glatte,
        orientierte Kurve $\gamma$ als $\text{Bild}(z) = z([a,b])$, durchlaufen von
        $a$ nach $b$. \\\\
        Notation: Dann ist $\gamma^-$ die in umgekehrter Richtung durchlaufene Kurve
        parametrisiert durch $z^-:[a,b]\to\C$ mit $z^-(t)=z(b+a-t)$.\\\\
        Die Punkte $z(a),z(b)$ heißen Endpunkte von $\gamma$.\\\\
        Die Kurve heißt geschlossen, wenn $z(a)=z(b)$.\\\\
        Die Kurve heißt einfach falls die Kurve geschlossen ist und injektiv in
        $(a,b)$.
    \end{definition}

    \begin{bemerkung}
        Ab jetzt wird jede Kurve (stückweise) glatt sein.
    \end{bemerkung}

    \begin{beispiel}
        ZEICHNUNG EINHEITSKREIS \\\\
        $[0,2\pi]\to\C, \quad t\mapsto e^{it}$ in positive Orientierung
        (das Innere liegt links) und \\
        $[0,2\pi]\to\C, \quad t\mapsto e^{it}$ in negative Orientierung
    \end{beispiel}

    \begin{definition}[Kurvenintegral]
        Sei $f$ stetig, $\gamma$ Kurve (mit Parametrisierung $z:[a,b]\to \C$),
        dann ist das Integral von $f$ entlang von $\gamma$ ist
        $$\int_\gamma f(z) dz = \int_a^b f(z(t))z'(t)\,dt $$
        falls $\gamma$ glatt. Falls $\gamma $ stückweise glatt, dann
        $$\int_{\gamma} f(z)\, dz = \sum_{j=0}^{n-1}\int_{a_i}^{a_{i+1}} f(z(t)) z'(t)\, dt$$
    \end{definition}

    \begin{bemerkung}
        Äquivalente Parametrisierungen ergeben das gleiche Kurvenintegral.
        $$\int_\gamma f(z)\, dz = \int_a^b f(z(t))z'(t)\, dt = \int_c^d f(z(t(s)))
        z'(t(s)) t'(s)\, ds = \int_c^d f(\tilde{z}(s))\tilde{z}'(s)\, ds$$
        Da $\tilde(s) = z(t(s))$ und $\tilde{z}'(s)=z'(t(s)) t'(s)$ mit
        Umparametrisierung $t= t(s)$.
    \end{bemerkung}

    \begin{definition}
        Die Länge von $\gamma$ ist
        $$L(\gamma):= \int_a^b |z'(t)|\, dt$$
        ÜBUNG: $L$ hängt nicht von der Parametrisierung ab!
    \end{definition}

    \begin{proposition}
        $\alpha,\beta\in \C$
        \begin{enumerate}[label = (\roman{enumi})]
            \item $\int_\gamma \alpha f(z)+\beta g(z)\, dz = \alpha \int_\gamma f(z)
        \,dz + \beta \int_\gamma g(z)\, dz$
            \item $\int_{\gamma^-} f(z)\,dz = -\int_\gamma f(z) \, dz$
            \item $\left|\int_\gamma f(z)\, dz\right|\leq \sup_{z\in\gamma}|f(z)|L(\gamma)$
        \end{enumerate}
    \end{proposition}

    \begin{proof}
        \begin{enumerate }[label = (\roman{enumi})]
            \item klar
            \item klar
            \item $\left|\int_\gamma f(z)\, dz\right|=\left|\int_a^b f(z(t))z'(t)
            \,dt\right|\leq \int_a^b |f(z(t))||z'(t)|\,dt
            \leq \sup_{z\in\gamma}|f(z)|\int_a^b |z'(t)|$
        \end{enumerate }
        $\quad$
    \end{proof}

    \underline{Integrale vs. Ableitungen}

    \begin{definition}
        Sei $f:\Omega\to\C$, $\Omega$ offen. Dann heißt $F:\Omega\to\C$ Stammfunktion
        von $f$ falls $F$ holomorph mit $F'(z) = f(Z)$.
    \end{definition}

    \begin{satz}
        Sei $f:\Omega\to \C$ stetig mit Stammfunktion $F$, $\gamma$ Kurve in
        $\Omega$ von $w_1$ nach $w_2$. Dann gilt:
        $$\int_\gamma f(z)\, dz = F(w_2)-F(w_1)$$
    \end{satz}

    \begin{proof}
        Sei $\gamma$ glatt, parametrisiert durch $z:[a,b]\to\C$
        $$\begin{aligned}
              \int_\gamma f(z)\, dz &= \int_a^b f(z(t))z'(t)\, dt\\& =
              \int_a^b F'(z(t)) z'(t) \,dt\\& = \int_a^b \frac{d}{dt}[F(z(t))]\, dt \\&=
              F(z(b))-F(z(a)) = F(w_2)-F(w_1)
        \end{aligned}$$
        Stückweise glatt funktioniert genau gleich, nur, dass wir eine Teleskopsumme
        erhalten.
    \end{proof}

    \begin{korollar}
        Falls außerdem $\gamma$ geschlossen mit den Voraussetzungen von oben,
        dann gilt
        $$\int_\gamma f(z)\, dz = 0$$
    \end{korollar}

    \begin{beispiel}
        Wir betrachten $\C\setminus \{0\}\to\C, z\mapsto \frac{1}{z}$ und
        untersuchen die Existenz einer Stammfunktion. \\\\
        Dafür integrieren wir über den Einheitskreis $\gamma :[0,2\pi]\to \C,
        t\mapsto e^{it}$. Dann gilt:
        $$\int_\gamma \frac{1}{z}\, dz = \int_0^{2\pi}\frac{1}{e^{it}}ie^{it} dt =
        \int_0^{2\pi}i \, dt = 2\pi i\neq 0$$
        Also gibt es keine Stammfunktion.
    \end{beispiel}

    \begin{korollar}
        Sei $f$ holomorph auf $\Omega$ offen und zusammenhängend, falls
        $f'=0\Rightarrow f$ konstant.
    \end{korollar}

    \begin{proof}
        Wir zeigen, dass für alle $w_1,w_2$ die Werte gleich sind. Dafür finden
        wir eine Kurve $\gamma$ (parametrisiert durch $z:[a,b]\to\Omega$ mit
        $z(a)=w_1, z(b)=w_2$) von $w_1$ nach $w_2$, damit gilt:
        $$f(w_2)-f(w_1) = \int_\gamma f'(z)\, dz = \int_\gamma 0\,dz = 0$$
        also sind die Funktionswerte gleich.
    \end{proof}

    \begin{definition}
        $\Omega\subset\C$ ist ein Gebiet, falls $\Omega$ offen und zusammenhängend.
    \end{definition}

    \chapter{Satz von Cauchy und Folgerungen}
    \begin{bemerkung}
        Informelle besatz der Satz von Cauchy: \\
        Sei $\Omega$ offen, geschlossene Kurve $\gamma\in\Omega$ deren Inneres
        auch in $\Omega$ liegt. Dann gilt
        $$0 = \int_\gamma f(z)\, dz$$
        für jede holomorphe Funktion.
    \end{bemerkung}

    \begin{satz}[Satz von Goursat (?)]
        Sei $\Omega\subset \C$ offen, $T\subset \Omega$ ein Dreieck dessen Inneres
        $D$ in $\Omega$ liegt. \\\\
        Falls $f:\Omega\to\C$ holomorph, dann
        $$\int_T f(z)\, dz = 0$$
    \end{satz}

    \begin{proof}
        Wir konstruieren eine Folge von Dreiecken durch sukzessives Halbieren der
        Seiten. (ZEICHNUNG)
        $$\int_{T^{(0)}} f(z)\, dz = \int_{T^{(1)}_1} f(z)\, dz+
        \int_{T^{(1)}_2} f(z)\, dz+\int_{T^{(1)}_3} f(z)\, dz+\int_{T^{(1)}_4} f(z)\, dz$$
        Damit wissen wir $\exists 1\leq  j\leq 4:$
        $$\left|\int_{T^{(0)}} f(z)\,dz\right|\leq
        4\left|\int_{T^{(1)}_j} f(z)\,dz\right|$$
        Umbenennung $T^{(1)} := T^{(1)}_j$, wieder kann man die Seiten halbieren und
        wieder ein $T^{(2)}$ finden. Durch Iteration erhalten wir so eine Folge von
        Dreiecken $T^{(0)},T^{(1)},T^{(2)},\dots$ mit Innerem
        $D^{(0)}\supset D^{(1)}\supset D^{(2)}\supset \dots$
        mit Durchmesser von $D^{(n)}=: d^n \leq \frac{1}{2}d^{(n-1)}\leq 2^{-n}d^{(0)}$
        und Umfang von $T^{(n)} =: p^{(n)}\leq 2^{-n}p^{(0)}$.
        $$\left|\int_{T^{(0)}} f(z)\,dz\right|\leq
        4^n\left|\int_{T^{(n)}} f(z)\,dz\right|$$
        Wir nehmen das Innere kompakt und finden damit:
        $$D^{(n)}\text{ kompakt}\Rightarrow \exists ! z_0\in \bigcap_{n\geq 1} D^{(n)}$$
        Schreibe $f(z) = f(z_0) + f'(z_0)(z-z_0) + R(z)(z-z_0)$ mit $R(z)\to 0$ für
        $z\to z_0$.\\\\
        Da $f(z_0)$ und $f'(z_0)(z-z_0)$ Stammfunktionen haben, gilt
        $$\int_{T^{(n)}}f(z)\, dz = \int_{T^{(n)}}R(z)(z-z_0)\,dz$$
        Daraus folgt:
        $$\left|\int_{T^{(0)}} f(z)\,dz\right|\leq
        4^n\left|\int_{T^{(n)}} f(z)\,dz\right| \leq
        \left|\int_{T^{(n)}}R(z)(z-z_0)\,dz\right\leq 4^n \sup_{z\in D^{(n)}}|R(z)|
        d^{(n)} p^{(n)}\leq 4^n \sup_{z\in D^{(n)}}|R(z)| 2^{-n} d^{(0)}2^{-n}p^{(0)}
        \leq \sup_{z\in D^{(n)}}|R(z)| d^{(0)}p^{(0)}\to 0(\quad \text{für  }n\to\infty)$$
    \end{proof}

    \section{Pausiert solange er "Complex Analysis, Stein, Shakarchi folgt!}









\end{document}