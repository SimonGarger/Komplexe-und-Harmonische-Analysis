\documentclass[a4paper, 12pt]{article}          % Paper and font size % Language setting
% Set margins
\usepackage[top=2.5cm,bottom=2.5cm,left=2cm,right=2cm,marginparwidth=1.75cm]{geometry}

% Load useful packages
\usepackage{graphicx}                           % Graphics
\usepackage{amsmath}                            % Math
\usepackage{xcolor}                             % Link color
\definecolor{custom-blue}{RGB}{0,99,166}
\usepackage[T1]{fontenc}                        % Special characters
\usepackage{fancyhdr}                           % Load header, footer package
\usepackage[ngerman]{babel}                     % Sprache


\PassOptionsToPackage{dvipsnames}{xcolor}       % Packages Anton
\usepackage{tikz}
\usepackage{amsfonts}
\usepackage{enumitem}
\usepackage{amssymb}
\usepackage{marvosym}
\usepackage{mathtools}
\usepackage{empheq}
\usepackage{cancel}
\usepackage{harpoon}
\usetikzlibrary{graphs,tikzmark,calc,arrows,arrows.meta,angles,math,decorations.markings}
\usepackage{pgfplots}
\usepackage{booktabs}
\usepackage{framed}
\usepackage[hyperref,amsmath,thmmarks,framed]{ntheorem}
\usepackage[colorlinks=true, linkcolor=magenta, psdextra, pdfencoding=auto]{hyperref}
\usepackage[capitalize,nameinlink]{cleveref}
\usepackage{tcolorbox}
\usepackage{lmodern}
\usepackage{faktor}
\usepackage{mathpunctspace}


\usepackage{lastpage}                           % Footer note
\usepackage{lipsum}
\usepackage{amsfonts}

\pagestyle{myheadings}                          % Own header
\pagestyle{fancy}                               % Own style
\fancyhf{}                                      % Clear header, footer

\setlength{\headheight}{30pt}                   % Set header hight
\renewcommand{\headrulewidth}{0.5pt}            % Top line
\renewcommand{\footrulewidth}{0.5pt}            % Bottom line
\setlength{\headsep}{1cm}

\fancyhead[L]{\includegraphics[width=3cm]{Uni_Logo_2016.jpg}} % Header left
\fancyhead[C]{}                                 % Header center
\fancyhead[R]{Komplexe und Harmonische Analysis}       % Header right
\fancyfoot[L]{}                                 % Footer left
\fancyfoot[C]{}                                 % Footer center
\fancyfoot[R]{\thepage/\pageref{LastPage}}      % Footer right

% Environments
% \newtheorem*{proof}{\textit{Beweis:}}
\DeclareRobustCommand{\qed}{%
    \ifmmode % if math mode, assume display: omit penalty etc.
    \else \leavevmode\unskip\penalty9999 \hbox{}\nobreak\hfill
    \fi
    \quad\hbox{\qedsymbol}}
\newcommand{\openbox}{\leavevmode
\hbox to.77778em{%
    \hfil\vrule
    \vbox to.675em{\hrule width.6em\vfil\hrule}%
    \vrule\hfil}}
\newcommand{\qedsymbol}{\openbox}
\newenvironment{proof}[1][\proofname]{\par
\normalfont\trivlist
\item[\hskip\labelsep\itshape
    #1.]\ignorespaces
}{%
\qed\endtrivlist
}

\newenvironment{bemerkung}[1][\textit{Bemerkung}]{\par
\normalfont\trivlist
\item[\hskip\labelsep\itshape
    #1.]\ignorespaces
}{%
\qed\endtrivlist
}

\newenvironment{beispiel}[1][\textit{Beispiel}]{\par
\normalfont\trivlist
\item[\hskip\labelsep\itshape
    #1.]\ignorespaces
}{%
    \qed\endtrivlist
}

\newcommand{\proofname}{Proof}
\makeatother

\definecolor{blau}{HTML}{29b0c2}
\definecolor{rosa}{HTML}{c48494}
\definecolor{weiss}{HTML}{FFFFFF}

\theoremstyle{break}
\theoremseparator{:\smallskip}
\theoremindent=1em
\theoremheaderfont{\kern-1em\normalfont\bfseries}
\theorembodyfont{\normalfont}
\theoreminframepreskip{0em}
\theoreminframepostskip{0em}
\theoremsymbol{}
\newtcbox{\theoremBox}{colback=rosa!17,colframe=rosa!87,boxsep=0pt,left=7pt,right=7pt,top=7pt,bottom=7pt}
\def\theoremframecommand{\theoremBox}

\newshadedtheorem{theo}{Theorem}[section]

\newshadedtheorem{satz}[theo]{Satz}
\newcommand{\satzautorefname}{Satz}
\theoremstyle{nonumberbreak}
\newshadedtheorem{nonumbersatz}{Satz}
\theoremstyle{break}
\newshadedtheorem{lemma}[theo]{Lemma}
\newcommand{\lemmaautorefname}{Lemma}
\newshadedtheorem{korollar}[theo]{Korollar}
\newcommand{\korollarautorefname}{Korollar}
\newshadedtheorem{folgerung}[theo]{Folgerung}
\newcommand{\folgerungautorefname}{Folgerung}
\newshadedtheorem{proposition}[theo]{Proposition}
\newcommand{\propositionautorefname}{Proposition}
\newtcbox{\definBox}{colback=blau!17,colframe=blau!94,boxsep=0pt,left=7pt,right=7pt,top=7pt,bottom=7pt}
\def\theoremframecommand{\definBox}
\newshadedtheorem{definition}[theo]{Definition}
\newcommand{\definautorefname}{Definition}
\newtcbox{\miscBox}{colback=gray!17,colframe=gray!80}

\def\theoremframecommand{\miscBox}
\newshadedtheorem{bemerkung}[theo]{Bemerkung}
\newcommand{\bemerkungautorefname}{Bemerkung}
\newshadedtheorem{beispiel}[theo]{Beispiel}

\newtcbox{\warnBox}{colback=red!17,colframe=red!80}
\def\theoremframecommand{\warnBox}
\theoremstyle{nonumberbreak}

\hypersetup{colorlinks=true, allcolors=custom-black}



\newcommand{\chapter}[1]{\pagebreak \huge\textbf{\vskip 2cm #1} \normalsize}
\newcommand{\N}{\mathbb{N}}
\newcommand{\Z}{\mathbb{Z}}
\newcommand{\R}{\mathbb{R}}
\newcommand{\C}{\mathbb{Ce}}
\renewcommand{\Re}{\mathfrak{Re}}
\renewcommand{\Im}{\mathfrak{Im}}


%%% Begin document %%%
\begin{document}
\begin{figure}                                  % Include Logo
    \flushleft
\includegraphics[width=0.6\textwidth]{Uni_Logo_2016.jpg}
\end{figure}

% Set title, author, date
\title{Komplexe und Harmonische Analysis WS2023}
\author{Simon Garger - simon.garger@gmail.com}
\date{\today{}, Wien}

\maketitle
\thispagestyle{empty}
\pagebreak% Include contents
\tableofcontents
\pagebreak


\chapter{Komplexe Analysis}
\\\\
    \begin{bemerkung}
        Im Folgenden wird immer $z_k=x_k+iy_k=r_k\cdot e^{i\theta_k}$ gelten. Außerdem wird im Allgemeinen
        $\Omega$ eine Teilmenge von $\C$ sein.
    \end{bemerkung}

\section{Grundlagen}

    \begin{definition}[Komplexe Zahlen]
        Wir definieren den komplexen Zahlenkörper als die Menge
        $$\C:= \{z=x+iy|x,y\in \R\}$$
        Dabei bezeichnet $i$ die komplexe Einheit mit $i^2=-1$.
        Man nennt $x=\Re(z)$ den Realteil und $y=\Im(z)$ den Imaginärteil.
        Der Betrag einer komplexen Zahl wird geschrieben durch $|z| = \sqrt{x^2+y^2}$.
        Wir definieren $\bar{z}:= x-iy$ als das komplex Konjugierte von $z$.
    \end{definition}

    \begin{proposition}
        Die komplexen Zahlen erfüllen folgende Rechenregeln:
        \begin{enumerate}
            \item $z_1+z_2 = (x_1+x_2)+i(y_1+y_2)$ \\(Interpretation: Vektoraddition)
            \item $z_{1}z_2 = (x_1x_2-y_1y_2)+i(x_1y_2+x_2y_1)=r_{1}r_2\cdot e^{i(\theta_1 +\theta_2)}$
            \\(Interpretation: Streckung und Drehung)
            \item $|z_1+z_2|\leq |z_1|+|z_2|$
            \item $|z|^2=z\bar{z}\Rightarrow \dfrac{1}{z} = \dfrac{\bar{z}}{|z|^2}$
        \end{enumerate}
    \end{proposition}

    \begin{definition}
        Da $\C\cong\R^2$ gilt, können wir die topologischen Eigenschaften von $\R^2$ in $\C$ übertragen
        und werden auch im Weiteren öfter die komplexen Zahlen mit der zweidimensionalen Zahlenebene
        identifizieren. Weiter werden wir eine Teilmenge von $\C$ meist mit $\Omega$ bezeichnen.
        Wir erhalten damit:
        \begin{enumerate}
            \item Die komplexen Zahlen sind vollständig, es konvergiert also jede Cauchy-Folge.
            \item Eine Folge konvergiert in $\C$ genau dann, wenn Real- und Imaginärteil konvergieren und genau
            dann, wenn es eine Cauchy-Folge ist.
            \item Das Innere von $\Omega$ ist $\Omega^\circ := \{a|\exists B_r: a\in B_r\subset \Omega\}$
            \item Wir nennen $\Omega$ kompakt, sofern die Menge beschränkt und abgeschlossen ist.
            \item Wir nennen $\Omega$ offen (abgeschlossen) zusammenhängend, wenn für $\Omega_1,\Omega_2$ offen
            (abgeschlossen) gilt:
            $$\Omega =\Omega_1\cup\Omega_2\quad\land\quad \emptyset = \Omega_1\cap \Omega_2\quad\Rightarrow
            \quad \Omega_1 = \emptyset\quad\lor \quad \Omega_2=\emptyset$$
            \item Wir nennen eine Funktion stetig, wenn gilt:
            $$\lim_{n\to \infty} z_n = z\quad \Rightarrow \quad \lim_{n\to \infty} f(z_n) = f(z)$$
        \end{enumerate}
    \end{definition}


\section{Holomorphie}

    \begin{definition}
        Sei $z_0\in \Omega$, dann nennen wir $f$ holomorph ("komplex differenzierbar") in $z_0$ falls
        $$\lim_{h\to 0}\dfrac{f(z_0+h)-f(z)}{h}=f'(z_0)\in \C$$
        existiert. \\\\
        Eine Funktion heißt auf $\Omega$ holomorph, wenn sie in jedem Punkt $z\in \Omega$ holomorph ist oder
        einfach "ganz", falls sie für alle $z\in \C$ holomorph ist.
    \end{definition}

    \begin{beispiel}
        Beispiele für holomorphe Funktionen sind:
        \begin{itemize}
            \item Konstante Funktionen
            \item Potenzfunktionen
            \item Polynomfunktionen
            \item Potenzreihen
        \end{itemize}
    \end{beispiel}

    \begin{proposition}[Cauchy-Riemann Gleichungen]
        Sei $f$ eine auf $\Omega$ holomorphe Funktion. Identifizieren wir nun $f(z) = u(z)+iv(z)$, wobei
        $u,v:\C\cong\R^2\to \R$ sind, so gelten folgende Beziehungen:
        $$\begin{aligned}
              \dfrac{\partial u}{\partial x}(x,y) &= \dfrac{\partial v}{\partial y}(x,y)\\
              \dfrac{\partial u}{\partial y}(x,y) &= -\dfrac{\partial v}{\partial x}(x,y)
        \end{aligned}$$
    \end{proposition}

    \begin{proof}
        Wir definieren für ein beliebiges $x+iy=z_0\in \Omega$:
        $$\lim_{h\to 0}\dfrac{f(z_0+h)-f(z)}{h}=f'(z_0)=: a+ib$$
        Nun differenzieren wir nur partiell (der obige Limes beschreibt ja eine beliebige Nullfolge, also
        können wir auch die entlang der Achsen betrachten), indem wir die obige $u,v$ Identifikation und
        $h=(h_1,h_2)$ verwenden:
        $$\begin{aligned}
              a+ib &= \lim_{h_1 \to 0}\dfrac{f(z_0+h_1)-f(z)}{h_1} \\
              &=\lim_{h_1 \to 0}\dfrac{u(z_0+h_1)+iv(z_0+h_1)-u(z_0)-iv(z_0)}{h_1}\\
              &=\lim_{h_1 \to 0}\dfrac{u(z_0+h_1)-u(z_0)}{h_1} +\lim_{h_1 \to 0}\dfrac{i(v(z_0+h_1)-iv(z_0))}{h_1}\\
              &=\dfrac{\partial u}{\partial x}(x,y) +i\dfrac{\partial v}{\partial x}(x,y)
        \end{aligned}$$
        Genauso finden wir:
        $$\begin{aligned}
              a+ib &= \lim_{h_2 \to 0}\dfrac{f(z_0+ih_2)-f(z)}{ih_2} \\
              &=\lim_{h_2 \to 0}\dfrac{u(z_0+ih_2)+iv(z_0+ih_2)-u(z_0)-iv(z_0)}{ih_2}\\
              &=\lim_{h_2 \to 0}\dfrac{u(z_0+ih_2)-u(z_0)}{ih_2} +\lim_{h_2 \to 0}\dfrac{i(v(z_0+ih_2)-v(z_0))}{ih_2}\\
              &=\frac{1}{i}\dfrac{\partial u}{\partial y}(x,y) +\dfrac{\partial v}{\partial y}(x,y)
        \end{aligned}$$
        Damit folgt: 
        $$\begin{aligned}
              &a = \dfrac{\partial u}{\partial x}(x,y) = \dfrac{\partial v}{\partial y}(x,y)\\
              &b = \dfrac{\partial v}{\partial x}(x,y) = -\frac{\partial u}{\partial y}(x,y)
        \end{aligned}$$
    \end{proof}









\end{document}